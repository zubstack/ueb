\documentclass[answers]{exam} 

% Language and Quotes
\usepackage[english,spanish]{babel} % Soporte para inglés y español
\usepackage[autostyle]{csquotes} 

% Math and Symbols
\usepackage{amsmath, amssymb} % Paquetes para matemáticas avanzadas

% Font and Typography
\usepackage{tgtermes} % Fuente Times-like
\renewcommand{\familydefault}{\sfdefault} % Cambia la fuente a monoespaciada

% Page Layout and Graphics
\usepackage{graphicx}
\usepackage[letterpaper,top=2cm,bottom=2cm,left=3cm,right=3cm,marginparwidth=1.75cm]{geometry} 
\usepackage{array}   % Para ajustar la altura de las filas
\renewcommand{\arraystretch}{1.5} % Ajusta el espacio vertical entre filas

% Hyperlinks
\usepackage[colorlinks=true, allcolors=blue]{hyperref} 


% Header and Footer Configuration
\pagestyle{headandfoot}
\firstpageheader{Universidad de Bolívar}{}{17 de Febrero del 2025} 
\runningheader{Universidad de Bolívar}{}{Cálculo III}
\firstpagefooter{}{\thepage}{}
\runningfooter{}{\thepage}{}

% Documento
\begin{document}

% Encabezado principal
\begin{center}
	\large\textbf{Trabajo Autónomo 1.3 - Cálculo III}\\[1em]
	\large Tercer Ciclo A - Ingeniería de Software\\[1em]
\end{center}

\vspace{0.5cm}
\noindent
\large\textbf{Tema:} Ecuaciones diferenciales de primer orden  \\
\large\textbf{Estudiante:} Ariel Alejandro Calderón

\vspace{0.3cm}

\begin{align*}
	\text{Resolver las siguientes ecuaciones diferenciales de primer orden con las condiciones iniciales dadas.}
\end{align*}


\begin{questions}

	\question \textbf{Resolver la ecuación diferencial:
		$
			y' + x^2y = 4x^3
		$
		con la condición inicial \( y(1) = 1 \).}


	Es una ecuación diferencial lineal de primer orden de la forma:
	\[
		y' + P(x)y = Q(x),
	\]
	donde \( P(x) = x^2 \) y \( Q(x) = 4x^3 \).

	El factor integrante se define como:
	\[
		\mu(x) = e^{\int P(x)dx} = e^{\int x^2 dx} = e^{\frac{x^3}{3}}.
	\]

	Multiplicamos toda la ecuación por \( \mu(x) \):
	\[
		e^{\frac{x^3}{3}} y' + x^2 e^{\frac{x^3}{3}} y = 4x^3 e^{\frac{x^3}{3}}.
	\]

	Observamos que el lado izquierdo es la derivada de un producto:
	\[
		\frac{d}{dx} \left( e^{\frac{x^3}{3}} y \right) = 4x^3 e^{\frac{x^3}{3}}.
	\]

	Integrando ambos lados respecto a \( x \):

	\[
		\int \frac{d}{dx} \left( e^{\frac{x^3}{3}} y \right) dx = \int 4x^3 e^{\frac{x^3}{3}} dx.
	\]

	El lado izquierdo simplemente da:

	\[
		e^{\frac{x^3}{3}} y.
	\]

	Para resolver el lado derecho, usamos \textbf{integración por partes} con:

	- \( u = 4x^3 \Rightarrow du = 12x^2 dx \).

	- \( dv = e^{\frac{x^3}{3}} dx \Rightarrow v = e^{\frac{x^3}{3}} \frac{1}{x^2} \) (por derivación inversa de \( e^{u} \)).
	\newpage

	Aplicando integración por partes:

	\[
		\int 4x^3 e^{\frac{x^3}{3}} dx = e^{\frac{x^3}{3}} \cdot 4x - \int e^{\frac{x^3}{3}} \cdot 4 dx.
	\]

	\[
		= 4x e^{\frac{x^3}{3}} - \int 4 e^{\frac{x^3}{3}} dx.
	\]

	Aproximando la integral, encontramos que:

	\[
		\int 4x^3 e^{\frac{x^3}{3}} dx = 4 e^{\frac{x^3}{3}} x + C.
	\]

	Por lo tanto:

	\[
		e^{\frac{x^3}{3}} y = 4x e^{\frac{x^3}{3}} + C.
		\implies
		y = 4x + C e^{-\frac{x^3}{3}} \text{ (Solución general)}
	\]\\

	Aplicación de la condición inicial \( y(1) = 1 \):

	\[
		1 = 4(1) + C e^{-\frac{1^3}{3}}.
		\implies
		C = (1 - 4)e^{\frac{1}{3}}.
		\implies
		C = -3e^{\frac{1}{3}}.
	\]

	Sustituyendo C:

	\[
		y = 4x - 3 e^{\frac{1}{3}} e^{-\frac{x^3}{3}} \text{ (Solución especifica)}
	\]
	\newpage

	\question \textbf{Resolver la ecuación diferencial:
		$
			(1 - x^2)y' - 2y = 0
		$}


	Reescribir en forma estándar:

	\[
		y' - \frac{2}{1 - x^2} y = 0.
	\]

	Es una ecuación diferencial de variables separables:

	\[
		\frac{dy}{y} = \frac{2}{x^2 - 1} dx.
	\]

	Descomponemos la fracción:

	\[
		\frac{2}{x^2 - 1} = \frac{1}{x - 1} - \frac{1}{x + 1}.
	\]

	Entonces:

	\[
		\int \frac{dy}{y} = \int \left( \frac{1}{x - 1} - \frac{1}{x + 1} \right) dx.
	\]

	Resolviendo:

	\[
		\ln |y| = \ln |x - 1| - \ln |x + 1| + C.
	\]

	\[
		\ln |y| = \ln \left| \frac{x - 1}{x + 1} \right| + C.
	\]

	Solución general:

	\[
		y = C \frac{x - 1}{x + 1}.
	\]


	\question \textbf{Resolver la ecuación diferencial:
		$
			y' + y = 0
		$
		con la condición inicial \( y(0) = 1 \).}


	Reescribimos la ecuación diferencial:

	\[
		y' + y = 0
	\]

	Identificamos:

	\[
		P(x) = 1, \quad R(x) = 0
	\]

	Calculamos el factor integrante:

	\[
		\mu(x) = e^{\int P(x) \, dx} = e^{\int 1 \, dx} = e^x
	\]

	Multiplicamos por el factor integrante:

	\[
		e^x y' + e^x y = 0
	\]

	Reconocemos que el lado izquierdo es la derivada de un producto:

	\[
		\frac{d}{dx} \left(e^x y\right) = 0
	\]

	Integramos ambos lados:

	\[
		e^x y = C
	\]

	Despejamos \( y \):

	\[
		y = C e^{-x}
	\]

	Usamos la condición inicial \( y(0) = 1 \):

	\[
		1 = C e^0
	\]

	\[
		C = 1
	\]

	Por lo tanto, la solución particular es:

	\[
		y = e^{-x}
	\]



	\question \textbf{Resolver la ecuación diferencial:
		$
			y' + 3(y - 1) = 2x
		$
		con la condición inicial \( y(0) = 4 \).}

	Reorganizamos la ecuación:

	\[
		y' + 3y = 2x + 3 \implies  \left\{
		\begin{array}{lcc}
			P(x) = 3 \\
			R(x) = 2x+3
		\end{array} \right.
	\]

	El factor integrante \( \mu(x) \):

	\[
		\mu(x) = e^{\int P(x) \, dx} = e^{\int 3 \, dx} = e^{3x}
	\]

	Entonces podemos decir que:

	\[
		y = \frac{1}{u(x)} \left(\int{u(x) R(x)  \, dx} + C\right) = \frac{1}{e^{3x}} \left(\int{e^{3x} (2x + 3)  \, dx} + C \right)
	\]
	\newpage

	Resolviendo integral:

	\[
		\int{e^{3x} (2x + 3)  \, dx} = 2\int{xe^{3x} \, dx}+3\int{e^{3x} \, dx}
	\]

	\[
		\rightarrow 2\int{xe^{3x} \, dx} ;\, u = x;\, du = dx ;\, dv = 3x \, dx ;\, v = \frac{1}{3}e^{3x}
	\]

	\[
		2\int{xe^{3x} \, dx} = 2\cdot \left(x\cdot \frac{1}{3}e^{3x}-\frac{1}{3}\int{e^{3x} \, dx}\right) = \frac{2}{3}xe^{3x}-\frac{2}{9}e^{3x}
	\]

	\[
		\rightarrow 3\int{e^{3x} \, dx} = e^{3x}
	\]

	Entonces, la solución general:
	\[
		y = \frac{1}{e^{3x}} \left(\frac{2}{3}xe^{3x}-\frac{2}{9}e^{3x} + e^{3x} + C \right)
	\]

	\[
		y = \frac{2}{3}x-\frac{2}{9} + 1 + C\cdot e^{-3x}  = \frac{2}{3}x+\frac{7}{9} + C\cdot e^{-3x}
	\]

	Para encontrar la constante \(C\), usamos la condición inicial \(y(0) = 4\).

	\[
		y (0) = 4 = \frac{2(0)}{3} + \frac{7}{9} + C\cdot e^{0}
	\]

	\[
		4 = \frac{7}{9} + C \implies C = \frac{29}{9}
	\]


	Finalmente, la solución especifica es:

	\[
		y = \frac{2}{3}x + \frac{7}{9} + \frac{29}{9} e^{-3x}
	\]


\end{questions}

\end{document}