\documentclass[12pt]{exam}

\usepackage{amsmath}

\begin{document}


\begin{questions}

\question Determine los valores de $m$ para los cuales las ecuaciones diferenciales dadas tienen una solución de la forma $y = e^{mx}$:
\begin{parts}
    \part $y'' + 10y' + 25y = 0$
    \part $y'' + 5y' + 25y = 0$
\end{parts}

\section*{Solución}

\begin{parts}
    \part Consideramos la ecuación característica asociada:
    \[
    m^2 + 10m + 25 = 0.
    \]
    Factorizando,
    \[
    (m+5)^2 = 0 \Rightarrow m = -5.
    \]

    \part La ecuación característica es:
    \[
    m^2 + 5m + 25 = 0.
    \]
    Resolviendo con la fórmula cuadrática:
    \[
    m = \frac{-5 \pm \sqrt{25 - 100}}{2} = \frac{-5 \pm \sqrt{-75}}{2} = \frac{-5 \pm 5i\sqrt{3}}{2}.
    \]
\end{parts}


\question Resuelva la ecuación diferencial: 
\[
y' + 3(y - 1) = 2x
\]
con $y(0) = 4$.

\section*{Solución}

La ecuación diferencial es:
\[
y' + 3(y - 1) = 2x.
\]
Reescribiéndola en la forma estándar:
\[
y' - 3y = 2x - 3.
\]
Este es un caso de ecuación diferencial lineal de primer orden, cuya solución general es:
\[
y(x) = e^{\int -3dx} \left( \int (2x - 3) e^{\int 3dx}dx + C \right).
\]
Calculando el factor integrante:
\[
\mu(x) = e^{-3x}.
\]
Multiplicando la ecuación por $\mu(x)$:
\[
e^{-3x} y' - 3e^{-3x} y = (2x - 3)e^{-3x}.
\]
Observamos que el lado izquierdo es la derivada de un producto:
\[
\frac{d}{dx} \left( e^{-3x} y \right) = (2x - 3)e^{-3x}.
\]
Integramos ambos lados:
\[
e^{-3x} y = \int (2x - 3)e^{-3x}dx.
\]
Resolviendo la integral por partes:
\[
\int 2x e^{-3x}dx = -\frac{2x}{3} e^{-3x} + \frac{2}{9} e^{-3x},
\]
\[
\int -3 e^{-3x}dx = e^{-3x}.
\]
Por lo tanto:
\[
e^{-3x} y = -\frac{2x}{3} e^{-3x} + \frac{2}{9} e^{-3x} + e^{-3x} C.
\]
Multiplicando por $e^{3x}$:
\[
y = -\frac{2x}{3} + \frac{2}{9} + C e^{3x}.
\]
Aplicando la condición inicial $y(0) = 4$:
\[
4 = -\frac{2(0)}{3} + \frac{2}{9} + C e^{0}.
\]
\[
4 = \frac{2}{9} + C.
\]
\[
C = \frac{36}{9} - \frac{2}{9} = \frac{34}{9}.
\]
La solución final es:
\[
y(x) = -\frac{2x}{3} + \frac{2}{9} + \frac{34}{9} e^{3x}.
\]


\question Resuelva la ecuación diferencial de primer orden:
\[
x^2 + \sin{x} - (y^2 - \cos{y}) y' = 0
\]

\section*{Solución}

La ecuación diferencial dada es:
\[
x^2 + \sin{x} - (y^2 - \cos{y}) y' = 0.
\]
Reescribiéndola en la forma estándar:
\[
M(x, y) + N(x, y) y' = 0,
\]
donde:
\[
M(x, y) = x^2 + \sin{x}, \quad N(x, y) = -(y^2 - \cos{y}).
\]

Verificamos la condición de exactitud, calculando las derivadas parciales:
\[
\frac{\partial M}{\partial y} = 0, \quad \frac{\partial N}{\partial x} = 0.
\]
Como \(\frac{\partial M}{\partial y} = \frac{\partial N}{\partial x}\), la ecuación es exacta.

Ahora encontramos la función potencial \( F(x, y) \) tal que:
\[
\frac{\partial F}{\partial x} = M = x^2 + \sin{x}.
\]

Integramos con respecto a \(x\):
\[
F(x, y) = \int (x^2 + \sin{x})dx = \frac{x^3}{3} - \cos{x} + g(y).
\]

Ahora usamos la ecuación:
\[
\frac{\partial F}{\partial y} = N = -(y^2 - \cos{y}).
\]

Derivamos \(F(x, y)\) respecto a \(y\):
\[
\frac{\partial}{\partial y} \left( \frac{x^3}{3} - \cos{x} + g(y) \right) = g'(y).
\]

Igualando con \(N(x, y)\):
\[
g'(y) = -(y^2 - \cos{y}).
\]

Integrando con respecto a \(y\):
\[
g(y) = -\frac{y^3}{3} + y\sin{y} + C.
\]

Por lo tanto, la solución implícita de la ecuación diferencial es:
\[
\frac{x^3}{3} - \cos{x} - \frac{y^3}{3} + y\sin{y} = C.
\]


\question Resuelva la ecuación diferencial:
\[
y' - \frac{3y}{x} = \frac{1}{y^3}, \quad y(1) = 1
\]
\section*{Solución}

La ecuación diferencial dada es:
\[
y' - \frac{3y}{x} = \frac{1}{y^3}, \quad y(1) = 1.
\]

Reescribiéndola en la forma estándar:
\[
\frac{dy}{dx} = \frac{3y}{x} + \frac{1}{y^3}.
\]

Observamos que es una ecuación de variables separables. Reescribimos:
\[
y^3 dy = \left( 3y x + 1 \right) \frac{dx}{x}.
\]

Separando términos:
\[
y^3 dy = \frac{dx}{x} + 3y \, dx.
\]

Sin embargo, reorganizando mejor:
\[
(y^3 - 3y) dy = \frac{dx}{x}.
\]

Factorizamos el lado izquierdo:
\[
y(y^2 - 3) dy = \frac{dx}{x}.
\]

Ahora integramos ambos lados:
\[
\int \frac{y(y^2 - 3)}{1} dy = \int \frac{dx}{x}.
\]

Para el lado izquierdo, expandimos la integral:
\[
\int y^3 dy - 3 \int y dy.
\]

Resolvemos:
\[
\frac{y^4}{4} - \frac{3y^2}{2} = \ln{|x|} + C.
\]

Aplicamos la condición inicial \( y(1) = 1 \):
\[
\frac{1}{4} - \frac{3}{2} = \ln{1} + C.
\]

Dado que \( \ln{1} = 0 \), tenemos:
\[
C = \frac{1}{4} - \frac{3}{2} = \frac{1}{4} - \frac{6}{4} = -\frac{5}{4}.
\]

Por lo tanto, la solución implícita es:
\[
\frac{y^4}{4} - \frac{3y^2}{2} = \ln{|x|} - \frac{5}{4}.
\]


\question Una partícula se mueve a lo largo del eje $x$ de modo que su velocidad en cualquier tiempo $t \geq 0$ está dada por:
\[
v(t) = \frac{1}{t^2 + 1}.
\]
Asumiendo que inicialmente se encuentra en el origen, ¿en qué tiempo la partícula se ha desplazado 1m?

\section*{Solución}

La ecuación diferencial dada es:
\[
v(t) = \frac{1}{t^2 + 1}.
\]
Sabemos que la velocidad \(v(t)\) es la derivada de la posición \(x(t)\) con respecto al tiempo:
\[
v(t) = \frac{dx}{dt}.
\]
Por lo tanto, la ecuación es:
\[
\frac{dx}{dt} = \frac{1}{t^2 + 1}.
\]
Para encontrar la posición \(x(t)\), integramos ambos lados:
\[
x(t) = \int \frac{1}{t^2 + 1} dt.
\]
La integral de \(\frac{1}{t^2 + 1}\) es una función estándar:
\[
x(t) = \tan^{-1}(t) + C.
\]
Dado que la partícula comienza en el origen, tenemos la condición inicial \(x(0) = 0\). Sustituyendo en la ecuación:
\[
0 = \tan^{-1}(0) + C.
\]
Sabemos que \(\tan^{-1}(0) = 0\), por lo que:
\[
C = 0.
\]
Entonces, la posición \(x(t)\) es:
\[
x(t) = \tan^{-1}(t).
\]
Ahora, queremos saber en qué tiempo la partícula se ha desplazado 1 metro, es decir, cuando \(x(t) = 1\). Entonces, resolvemos:
\[
1 = \tan^{-1}(t).
\]
Despejamos \(t\):
\[
t = \tan(1).
\]
Por lo tanto, el tiempo en el que la partícula se ha desplazado 1 metro es:
\[
t = \tan(1) \approx 1.557.
\]


\end{questions}

\end{document}
