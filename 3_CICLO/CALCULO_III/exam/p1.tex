\documentclass[answers]{exam} 

% Language and Quotes
\usepackage[english,spanish]{babel} % Soporte para inglés y español
\usepackage[autostyle]{csquotes} 

% Math and Symbols
\usepackage{amsmath, amssymb} % Paquetes para matemáticas avanzadas

% Font and Typography
\usepackage{tgtermes} % Fuente Times-like
\renewcommand{\familydefault}{\ttdefault} % Cambia la fuente a monoespaciada

% Page Layout and Graphics
\usepackage{graphicx}
\usepackage[letterpaper,top=2cm,bottom=2cm,left=3cm,right=3cm,marginparwidth=1.75cm]{geometry} 
\usepackage{array}   % Para ajustar la altura de las filas
\renewcommand{\arraystretch}{1.5} % Ajusta el espacio vertical entre filas

% Hyperlinks
\usepackage[colorlinks=true, allcolors=blue]{hyperref} 

% Documento
\begin{document}

% Encabezado principal
\begin{center}
	\large\textbf{Prueba 1}
\end{center}

\begin{questions}

	\question \large\textbf{Compruebe que $y_1 = \sin x$ y $y_2=\dfrac{\sin 2x}{\sin x}$ son soluciones de $y''+y=0$:}

	Primero, verificamos para $y_1 = \sin x$:
	\[
		y_1' = \cos x \quad \text{y} \quad y_1'' = -\sin x
	\]
	Sustituyendo en la ecuación $y'' + y = 0$:
	\[
		y_1'' + y_1 = -\sin x + \sin x = 0 \quad \checkmark
	\]

	Ahora, para $y_2 = \dfrac{\sin 2x}{\sin x}$:

	\[
		y_2 = \frac{2\sin x \cos x}{\sin x} = 2\cos x
	\]

	\[
		y_2' = -2\sin x \quad \text{y} \quad y_2'' = -2\cos x
	\]

	\[
		y_2'' + y_2 = -2\cos x + 2\cos x = 0 \quad \checkmark
	\]

	\vspace{0.5cm}


	\question \large\textbf{Resolver por integración directa la siguiente ecuación diferencial:}

	\[y''' - 5x = 0\]



	1.  \[y'' = \int 5x \, dx = \dfrac{5x^2}{2} + C_1\]
	2.  \[y' = \int \left(\dfrac{5x^2}{2} + C_1\right) dx = \dfrac{5x^3}{6} + C_1x + C_2\]
	3. 	 \[y = \int \left(\dfrac{5x^3}{6} + C_1x + C_2\right) dx = \dfrac{5x^4}{24} + \dfrac{C_1x^2}{2} + C_2x + C_3\]

	Esta es la solución general de la ecuación diferencial.

	\vspace{0.5cm}

	\newpage
	\question \large\textbf{Aplicando la ley de voltajes de Kirchhoff, encontrar la corriente $i$ cuando $t=10$ segundos que circula por el circuito resistivo inductivo de la figura. Condición inicial: $i(0)=15$ Amp.}

	\[
		V_S = 110 \text{ Voltios}, \quad
		R = 10 \Omega, \quad
		L = 5 \text{ Henrios}, \quad
		V_R = iR, \quad
		V_L = L\dfrac{di}{dt}
	\]


	Aplicamos la Ley de Voltajes de Kirchhoff al circuito:
	\[
		V_s = V_R + V_L
	\]
	Donde:
	\[
		V_R = iR \quad \text{y} \quad V_L = L \frac{di}{dt}
	\]
	Sustituyendo los valores dados:
	\[
		110 = 10i + 5\frac{di}{dt}
	\]
	Reordenamos la ecuación diferencial:
	\[
		\frac{di}{dt} + 2i = 22
	\]

	Resolvemos la homogénea: $\frac{di_h}{dt} + 2i_h = 0$
	\[
		i_h(t) = Ce^{-2t}
	\]

	Buscamos una solución particular $i_p$:
	Suponemos $i_p = k$ (constante):
	\[
		2k = 22 \quad \Rightarrow \quad k = 11
	\]

	Solución general:
	\[
		i(t) = i_h + i_p = Ce^{-2t} + 11
	\]

	Usamos la condición inicial $i(0)=15$:
	\[
		15 = C + 11 \quad \Rightarrow \quad C = 4
	\]
	Así que:
	\[
		i(t) = 4e^{-2t} + 11
	\]

	Calculamos $i(10)$:
	\[
		i(10) = 4e^{-20} + 11 \approx 11 \quad (\text{ya que } e^{-20} \approx 0)
	\]

	Por lo tanto, la corriente en $t=10$ s es aproximadamente $11$ A.

	\vspace{0.5cm}

	\newpage

	\question \large\textbf{Resuelva por dos métodos la ecuación diferencial: $y' = x + y + 1$:}\\


	\textbf{Método 1: Factor integrante}
	Reescribimos:
	\[
		y' - y = x + 1 \implies
		\left\{
		\begin{array}{lcc}
			P(x) = -1 \\
			R(x) = x+1
		\end{array} \right.
	\]
	El factor integrante es:
	\[
		\mu(x) = e^{\int -1 \, dx} = e^{-x}
	\]

	Dada la forma de la ecuacion sabemos que:
	\[
		y = \frac{1}{u(x)} \left(\int{u(x) R(x)  \, dx} + C\right)
	\]
	\[
		y = e^x \cdot \int (x+1)e^{-x} \, dx = e^x \cdot \left(\int xe^{-x} \, dx + \int e^{-x} \, dx\right)
	\]

	Usando integración por partes, obtenemos:
	\[
		\int xe^{-x} \, dx = -xe^{-x} - e^{-x} + C
	\]
	\[
		\int e^{-x} \, dx = - e^{-x} + C
	\]
	Entonces:
	\[
		y = e^x \cdot (-xe^{-x} - 2e^{-x} + C)
	\]

	\[
		\boxed{y = Ce^x -x - 2}
	\]

	\textbf{Método 2: Separacion de variables}

	Sea $v = y + x + 1$, sabemos que si $y' = f(ax+by+c)$, entonces:
	\[
		\int \dfrac{dv}{b\cdot f(v)+a} = \int dx
	\]
	
	\[
		\int \dfrac{dv}{v+1} = x+C
	\]

	\[
		\ln |v+1|=x+C \implies v=e^{x+C}-1
	\]

	Volvemos a $y$:
	\[
		y + x + 1 = e^{x+C}-1 \implies \boxed{y = C_1 e^x - x - 2}, \quad C_1 = e^C
	\]

	\vspace{0.5cm}

\end{questions}

\end{document}