\documentclass[answers]{exam} 
\usepackage[english,spanish]{babel} % Soporte para inglés y español
\usepackage[autostyle]{csquotes} 
\usepackage{graphicx}
\usepackage[letterpaper,top=2cm,bottom=2cm,left=3cm,right=3cm,marginparwidth=1.75cm]{geometry} 
\usepackage{amsmath, amssymb} % Paquetes para matemáticas avanzadas
\usepackage[colorlinks=true, allcolors=blue]{hyperref} 
\usepackage{array}   % for adjusting row height
\renewcommand{\arraystretch}{1.5} % adjust the vertical spacing between rows
\renewcommand{\familydefault}{\sfdefault}

% Configuración de encabezado y pie de página
\pagestyle{headandfoot}
\firstpageheader{Universidad de Bolívar}{}{10 de Febrero del 2025} 
\runningheader{Universidad de Bolívar}{}{Cálculo III}
\firstpagefooter{}{\thepage}{}
\runningfooter{}{\thepage}{}

% Documento
\begin{document}

% Encabezado principal
\begin{center}
	\large\textbf{Trabajo Autónomo 1.2 - Cálculo III}\\[1em]
	\large Tercer Ciclo A - Ingeniería de Software\\[1em]
\end{center}
\vspace{0.5cm}
\noindent
\large\textbf{Tema:} Introducción a las Ecuaciones Diferenciales. \\
\large\textbf{Estudiante:} Ariel Alejandro Calderón
\vspace{0.5cm}


\begin{questions}

	\question\textbf{ Determine el tipo, orden y la linealidad de las siguientes ecuaciones diferenciales: }

	\begin{center}
		\renewcommand{\arraystretch}{1.5} % Aumenta la separación entre filas
		\begin{tabular}{|c|c|c|c|}
			\hline
			\textbf{Ecuación Diferencial}   & \textbf{Orden} & \textbf{Linealidad} & \textbf{Tipo} \\
			\hline
			\( y'' + 3y = 0 \)              & 2              & Lineal              & Homogénea     \\
			\hline
			\( y'' + 3y = 2x + 5 \)         & 2              & Lineal              & No homogénea  \\
			\hline
			\( y'' + 3yy' = 0 \)            & 2              & No lineal           & Homogénea     \\
			\hline
			\( y''' + 2(y')^2 + 3y = 5 \)   & 3              & No lineal           & No homogénea  \\
			\hline
			\( y'' + 3x^4 y = 0 \)          & 2              & Lineal              & Homogénea     \\
			\hline
			\( y' + 3xy^4 = e^{-2x} \)      & 1              & No lineal           & No homogénea  \\
			\hline
			\( y''' + y' + \sin(y) = 0.2 \) & 3              & No lineal           & No homogénea  \\
			\hline
		\end{tabular}
	\end{center}

	\vspace{2em}

	\question \textbf{Compruebe que \( y_1 = 3e^{-2x} \) es solución de la ecuación diferencial \( y'' - 4y = 0 \)}.\\


	Comprobamos si \( y_1 = 3e^{-2x} \) satisface la ecuación diferencial \( y'' - 4y = 0 \).

	\textbf{Derivadas de \( y_1 \)}:

	\[
		y_1 = 3e^{-2x}, \quad y_1' = -6e^{-2x}, \quad y_1'' = 12e^{-2x}
	\]

	\textbf{Sustitución en la ecuación diferencial}:

	\[
		y'' - 4y = 12e^{-2x} - 4(3e^{-2x})
	\]

	\[
		= 12e^{-2x} - 12e^{-2x} = 0
	\]

	\(\Rightarrow\) Por lo tanto, \( y_1 \) es solución.

	\newpage


	\question \textbf{Compruebe que \( y_1 = \sin x \) y $y_2 = \dfrac{\sin 2x}{\sin x}$ son soluciones de \( y'' + y = 0 \).}\\

	\textbf{Para \( y_1 = \sin x \):}

	\[
		y_1' = \cos x, \quad y_1'' = -\sin x
	\]

	Sustituyendo en la ecuación:

	\[
		y_1'' + y_1 = -\sin x + \sin x = 0
	\]

	\(\Rightarrow\) \( y_1 \) es solución.


	\textbf{Para \( y_2 = \frac{\sin 2x}{\sin x} \):}

	Reescribimos usando la identidad trigonométrica \( \sin 2x = 2\sin x \cos x \):

	\[
		y_2 = \frac{2\sin x \cos x}{\sin x} = 2\cos x
	\]

	\textbf{Derivadas:}

	\[
		y_2' = -2\sin x, \quad y_2'' = -2\cos x
	\]

	\textbf{Sustituyendo en la ecuación diferencial:}

	\[
		y_2'' + y_2 = (-2\cos x) + (2\cos x) = 0
	\]

	\(\Rightarrow\) \( y_2 \) es solución.

	\vspace{2em}
	\question \textbf{Resolver por integración directa las siguientes ecuaciones diferenciales:}\\
    
		\textbf{1. Resolver \( y'' = 0 \):}

		Integrando una vez:

		\[
			y' = C_1
		\]

		Integrando nuevamente:

		\[
			y = C_1 x + C_2
		\]

		\(\Rightarrow\) Solución general: \( y = C_1 x + C_2 \).

		\newpage

		\textbf{2. Resolver \( y'' - x = 0 \):}

		Reescribimos:

		\[
			y'' = x
		\]

		Integrando una vez:

		\[
			y' = \frac{x^2}{2} + C_1
		\]

		Integrando nuevamente:

		\[
			y = \frac{x^3}{6} + C_1 x + C_2
		\]
		\vspace{2em}

		\(\Rightarrow\) Solución general: \( y = \frac{x^3}{6} + C_1 x + C_2 \).

		\vspace{2em}

		\textbf{3. Resolver \( y''' - 5x = 0 \):}

		Reescribimos:

		\[
			y''' = 5x
		\]

		Integrando una vez:

		\[
			y'' = \frac{5x^2}{2} + C_1
		\]

		Integrando nuevamente:

		\[
			y' = \frac{5x^3}{6} + C_1 x + C_2
		\]

		Integrando una última vez:

		\[
			y = \frac{5x^4}{24} + \frac{C_1 x^2}{2} + C_2 x + C_3
		\]

		\vspace{2em}

		\(\Rightarrow\) Solución general: \( y = \frac{5x^4}{24} + \frac{C_1 x^2}{2} + C_2 x + C_3 \).




\end{questions}
\end{document}