\documentclass[answers]{exam} 
\usepackage[english,spanish]{babel} % Soporte para inglés y español
\usepackage[autostyle]{csquotes} 
\usepackage{graphicx}
\usepackage[letterpaper,top=2cm,bottom=2cm,left=2cm,right=2cm,marginparwidth=1.75cm]{geometry} 
\usepackage{amsmath, amssymb} % Paquetes para matemáticas avanzadas
\usepackage[colorlinks=true, allcolors=blue]{hyperref} 
\usepackage{array}   % for adjusting row height
\renewcommand{\arraystretch}{1.5} % adjust the vertical spacing between rows
\renewcommand{\familydefault}{\sfdefault}

\edef\svparindent{\the\parindent}
\newenvironment{specindent}
  {\par\everypar{\leftskip=\svparindent\relax}\parindent=0pt\relax\parskip=1ex}
  {\vspace{\parskip}\par}

% Configuración de encabezado y pie de página
% \pagestyle{headandfoot}
% \firstpageheader{Universidad de Bolívar}{}{fecha} 
% \runningheader{Universidad de Bolívar}{}{Cálculo III}
% \firstpagefooter{}{\thepage}{}
\runningfooter{}{}{}

% Documento
\begin{document}

% % Encabezado principal
% \begin{center}
% 	\large\textbf{Trabajo Autónomo ? - Cálculo III}\\[1em]
% 	\large Tercer Ciclo A - Ingeniería de Software\\[1em]
% \end{center}
% \vspace{0.5cm}
% \noindent
% \large\textbf{Tema:} tema \\
% \large\textbf{Estudiante:} Ariel Alejandro Calderón
% \vspace{0.5cm}
\section*{Problema 1}

\begin{specindent}
	El radio se desintegra con una velocidad proporcional a la cantidad presente. La vida media es de 1600 años, tiempo necesario para que se descomponga la mitad. Hallar la cantidad desintegrada en 100 años.

\end{specindent}
\vspace{0.5em}

\textbf{1. Analizar y comprender el problema presentado}

La desintegración radiactiva sigue un modelo exponencial donde la tasa de cambio de la cantidad de sustancia \( N \) es

proporcional a la cantidad presente:


\[
	\frac{dN}{dt} = -kN
\]

donde: \\

- \( N(t) \) es la cantidad de radio en el tiempo \( t \),

- \( k \) es la constante de desintegración,

- el signo negativo indica que la cantidad de material disminuye con el tiempo.

Se nos proporciona la vida media:

\[
	t_{1/2} = 1600 \text{ años}
\]

y queremos determinar la cantidad desintegrada en \( t = 100 \) años.

\vspace{0.3cm}
\textbf{2. Revisar o investigar las leyes físicas que presenta el problema}

La ecuación diferencial de desintegración radiactiva es:

\[
	N(t) = N_0 e^{-kt}
\]

donde:\\

- \( N_0 \) es la cantidad inicial de la sustancia radiactiva

- \( k \) es la constante de desintegración que depende de la vida media.

Sabemos que la relación entre la constante de desintegración \( k \) y la vida media es:

\[
	k = \frac{\ln 2}{t_{1/2}}
\]

\vspace{0.3cm}
\textbf{3. Modelar el problema como una ecuación diferencial con valores iniciales}

La ecuación diferencial es:

\[
	\frac{dN}{dt} = -kN
\]

y su solución es:

\[
	N(t) = N_0 e^{-kt}
\]

Para encontrar \( k \), usamos la vida media:

\[
	k = \frac{\ln 2}{1600}
	\implies
	k \approx \frac{0.693}{1600} \approx 0.000433
\]

\vspace{0.3cm}
\textbf{4. Encontrar la solución general de la ecuación diferencial}

La cantidad de radio en cualquier tiempo \( t \) está dada por:

\[
	N(t) = N_0 e^{-0.000433t}
\]

\vspace{0.3cm}
\textbf{5. Encontrar la solución específica}

Para hallar la cantidad desintegrada en \( t = 100 \) años, calculamos:

\[
	N(100) = N_0 e^{-0.000433(100)}
\]

\[N(100)= N_0 e^{-0.0433}\]

\[N(100) \approx N_0 (0.9576)\]

La cantidad desintegrada es:

\[
	\Delta N = N_0 - N(100)
\]

\[
	\Delta N = N_0 - N_0(0.9576)
\]

\[
	\Delta N = N_0 (1 - 0.9576)
\]

\[
	\Delta N = N_0 (0.0424)
\]

\vspace{0.3cm}
\textbf{6. Encontrar la solución del problema real}

El porcentaje de sustancia que se ha desintegrado en 100 años es:

\[
	4.24\% \text{ de la cantidad inicial}
\]

Por lo que, en términos absolutos:

\[
	\Delta N = 0.0424 N_0
\]


% \vspace{1cm}
\section*{Problema 2}

\begin{specindent}
	Suponga que una gota de un líquido se evapora con una velocidad proporcional a su superficie. Hallar el radio de la gota en función del tiempo.
\end{specindent}

\vspace{0.5em}

\textbf{1. Analizar y comprender el problema presentado}

Tenemos una gota de líquido que se evapora a una velocidad proporcional a su superficie.

Si suponemos que la gota es esférica, su volumen está dado por:

\[
	V = \frac{4}{3} \pi r^3
\]

y su superficie por:

\[
	S = 4 \pi r^2
\]

Dado que la evaporación implica la pérdida de volumen, podemos escribir la ecuación diferencial:

\[
	\frac{dV}{dt} = - k S
\]

donde:
- \( V \) es el volumen de la gota,
- \( S \) es la superficie de la gota,
- \( k \) es una constante de proporcionalidad positiva.

\vspace{0.3cm}
\textbf{2. Revisar o investigar las leyes físicas que presenta el problema}

Sustituyendo la expresión de \( V \) y \( S \):

\[
	\frac{d}{dt} \left( \frac{4}{3} \pi r^3 \right) = - k (4 \pi r^2)
\]

Calculando la derivada:

\[
	4 \pi r^2 \frac{dr}{dt} = - 4 \pi k r^2
\]

Simplificando:

\[
	\frac{dr}{dt} = - k
\]

\vspace{0.3cm}
\textbf{3. Modelar el problema como una ecuación diferencial con valores iniciales}

Tenemos la ecuación diferencial:

\[
	\frac{dr}{dt} = -k
\]

con la condición inicial:

\[
	r(0) = r_0
\]

donde \( r_0 \) es el radio inicial de la gota.

\vspace{0.3cm}
\textbf{4. Encontrar la solución general de la ecuación diferencial}

Separando variables:

\[
	dr = - k dt
\]

Integrando ambos lados:

\[
	r = - k t + C
\]

\vspace{0.3cm}
\textbf{5. Encontrar la solución específica}

Usando la condición inicial \( r(0) = r_0 \):

\[
	r_0 = - k (0) + C \Rightarrow C = r_0
\]

Por lo que la solución es:

\[
	r(t) = r_0 - k t
\]

\vspace{0.3cm}
\textbf{6. Encontrar la solución del problema real}

El radio de la gota en función del tiempo es:

\[
	r(t) = r_0 - k t
\]

Esta ecuación indica que la gota se evapora de manera lineal con el tiempo hasta que \( r(t) = 0 \), es decir, cuando:

\[
	t = \frac{r_0}{k}
\]

lo que significa que la gota desaparece completamente en \( t = r_0 / k \).



\end{document}