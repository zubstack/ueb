\documentclass[answers]{exam} 
\usepackage[english,spanish]{babel} % Soporte para inglés y español
\usepackage[autostyle]{csquotes} 
\usepackage{graphicx}
\usepackage[letterpaper,top=2cm,bottom=2cm,left=2cm,right=2cm,marginparwidth=1.75cm]{geometry} 
\usepackage{amsmath, amssymb} % Paquetes para matemáticas avanzadas
\usepackage[colorlinks=true, allcolors=blue]{hyperref} 
\usepackage{array}   % for adjusting row height
\renewcommand{\arraystretch}{1.5} % adjust the vertical spacing between rows
\renewcommand{\familydefault}{\sfdefault}

\edef\svparindent{\the\parindent}
\newenvironment{specindent}
  {\par\everypar{\leftskip=\svparindent\relax}\parindent=0pt\relax\parskip=1ex}
  {\vspace{\parskip}\par}

% Configuración de encabezado y pie de página
% \pagestyle{headandfoot}
% \firstpageheader{Universidad de Bolívar}{}{fecha} 
% \runningheader{Universidad de Bolívar}{}{Cálculo III}
% \firstpagefooter{}{\thepage}{}
\runningfooter{}{}{}

% Documento
\begin{document}

% % Encabezado principal
% \begin{center}
% 	\large\textbf{Trabajo Autónomo ? - Cálculo III}\\[1em]
% 	\large Tercer Ciclo A - Ingeniería de Software\\[1em]
% \end{center}
% \vspace{0.5cm}
% \noindent
% \large\textbf{Tema:} tema \\
% \large\textbf{Estudiante:} Ariel Alejandro Calderón
% \vspace{0.5cm}


\section*{Problema 1}

\begin{specindent}
	Una partícula P se mueve a lo largo del eje x de tal manera que su aceleración en cualquier tiempo $t \ge 0$, está dada por $a = 8 - 12 t$.
\end{specindent}

\begin{parts}

	\part Encuentre la posición x de la partícula, medida del origen 0 a cualquier tiempo t > 0, asumiendo que inicialmente está localizada en x = 2 y está viajando a una velocidad $V = -2$.
	\part Resuelva a) si solo se sabe que la partícula está
	localizada inicialmente en x = 2 y en x = 7, cuando t = 1.

\end{parts}
\vspace{0.5em}


\textbf{1. Analizar y comprender el problema presentado}

La partícula \( P \) se mueve a lo largo del eje \( x \) con una aceleración dependiente del tiempo:

\[
	a(t) = 8 - 12t
\]


\vspace{0.3cm}
\textbf{2. Revisar o investigar las leyes físicas que presenta el problema}

Sabemos que la aceleración es la derivada de la velocidad respecto al tiempo:

\[
	a(t) = \frac{dv}{dt}
\]

Asimismo, la velocidad es la derivada de la posición respecto al tiempo:

\[
	v(t) = \frac{dx}{dt}
\]

Para encontrar la posición \( x(t) \), integramos sucesivamente la aceleración.

\vspace{0.3cm}
\textbf{3. Modelar el problema como una ecuación diferencial con valores iniciales}

Dado que \( a(t) = \dfrac{dv}{dt} \), la ecuación diferencial a resolver es:

\[
	\frac{dv}{dt} = 8 - 12t
\]

Con la condición inicial \( v(0) = -2 \).

Luego, como \( v(t) = \dfrac{dx}{dt} \), obtenemos la ecuación:

\[
	\frac{dx}{dt} = v(t)
\]

Con la condición inicial \( x(0) = 2 \).

\vspace{0.3cm}
\textbf{4. Encontrar la solución general de la ecuación diferencial}

Para la ecuación de aceleración:

\[
	\int dv = \int (8 - 12t) dt
\]

\[
	v(t) = 8t - 6t^2 + C_1
\]

Para la ecuación de velocidad:

\[
	\int dx = \int (8t - 6t^2 + C_1) dt
\]

\[
	x(t) = 4t^2 - 2t^3 + C_1 t + C_2
\]

\newpage
\textbf{5. Encontrar la solución específica}

Usamos las condiciones iniciales:\\

1. \( v(0) = -2 \Rightarrow 8(0) - 6(0)^2 + C_1 = -2 \Rightarrow C_1 = -2 \)\\

2. \( x(0) = 2 \Rightarrow 4(0)^2 - 2(0)^3 -2(0) + C_2 = 2 \Rightarrow C_2 = 2 \)\\

Sustituyendo en la ecuación de posición:

\[
	x(t) = 4t^2 - 2t^3 - 2t + 2
\]


\textbf{6. Encontrar la solución del problema real}

\[
	x(t) = 4t^2 - 2t^3 - 2t + 2
\]

Para la segunda parte del problema, donde se da que \( x(0) = 2 \) y \( x(1) = 7 \):

\[
	4(1)^2 - 2(1)^3 - 2(1) + C_2 = 7
\]

Resolviendo para \( C_2 \):

\[
	4 - 2 - 2 + C_2 = 7 \implies C_2 = 7
\]

Por lo que la nueva ecuación de posición es:

\[
	x(t) = 4t^2 - 2t^3 - 2t + 7
\]

\textbf{Literal b)}: El problema enuncia que "la partícula está localizada inicialmente en x = 2 y en x = 7, cuando t = 1" lo cual es \textbf{imposible}, ya que un cuerpo no puede hallarse en dos posiciones distintas a la vez.

\section*{Problema 2}

\begin{specindent}
	Una partícula se mueve a lo largo del eje x, de tal manera que su velocidad es proporcional
	al producto de su posición instantánea x (medida de x = 0) y el tiempo t (medido de t = 0). Si
	la partícula está localizada en $x = 25 m$ cuando t = 0 seg. y $x = 12 m$ cuando t = 1seg, ¿dónde
	estará cuando t=2?
	
\end{specindent}

\vspace{1em}
\textbf{1. Analizar y comprender el problema presentado}

La partícula se mueve a lo largo del eje \( x \), y su velocidad \( v \) es proporcional al producto de su posición instantánea \( x \) y el tiempo \( t \):

\[
	v = k x t
\]

donde \( k \) es una constante de proporcionalidad.

Las condiciones iniciales son:\\
- \( x(0) = 25 \) m\\
- \( x(1) = 12 \) m

Nos piden encontrar \( x(2) \).

\newpage

\textbf{2. Revisar o investigar las leyes físicas que presenta el problema}

Sabemos que la velocidad es la derivada de la posición con respecto al tiempo:

\[
	v = \frac{dx}{dt}
\]

Sustituyendo la relación dada en el problema:

\[
	\frac{dx}{dt} = k x t
\]

Esta es una ecuación diferencial separable.

\vspace{0.3cm}
\textbf{3. Modelar el problema como una ecuación diferencial con valores iniciales}

La ecuación diferencial a resolver es:

\[
	\frac{dx}{dt} = k x t
\]

con las condiciones:

\[
	x(0) = 25, \quad x(1) = 12
\]

\vspace{0.3cm}
\textbf{4. Encontrar la solución general de la ecuación diferencial}

Reescribimos la ecuación en forma separable:

\[
	\frac{dx}{x} = k t dt
\]

Integramos ambos lados:

\[
	\int \frac{dx}{x} = \int k t dt
\]

Resolviendo:

\[
	\ln |x| = \frac{k t^2}{2} + C
\]

Despejando \( x \):

\[
	x(t) = e^{C} e^{\frac{k t^2}{2}}
\]

Definiendo \( e^C = C_1 \):

\[
	x(t) = C_1 e^{\frac{k t^2}{2}}
\]

\newpage
\textbf{5. Encontrar la solución específica}

Usamos la condición inicial \( x(0) = 25 \):

\[
	25 = C_1 e^{\frac{k(0)^2}{2}}
	\implies
	C_1 = 25
\]

Entonces,

\[
	x(t) = 25 e^{\frac{k t^2}{2}}
\]

Ahora usamos \( x(1) = 12 \) para encontrar \( k \):

\[
	12 = 25 e^{\frac{k (1)^2}{2}}
\]

\[
	\frac{12}{25} = e^{\frac{k}{2}}
\]

Tomamos logaritmo natural:

\[
	\ln \left(\frac{12}{25} \right) = \frac{k}{2}
\]

\[
	k = 2 \ln \left(\frac{12}{25} \right)
\]

\vspace{0.3cm}
\textbf{6. Encontrar la solución del problema real}

Ahora buscamos \( x(2) \):

\[
	x(2) = 25 e^{\frac{k(2)^2}{2}}
\]

Sustituyendo \( k = 2 \ln \left(\frac{12}{25} \right) \):

\[
	x(2) = 25 e^{\frac{2 \ln(12/25) \cdot 4}{2}}
\]

\[
	x(2) = 25 e^{4 \ln(12/25)}
\]

\[
	x(2) = 25 \left( e^{\ln(12/25)} \right)^4
\]

\[
	x(2) = 25 \left(\frac{12}{25} \right)^4
\]

Resolviendo:

\[
	x(2) = 25 \times \left(\frac{20736}{390625} \right)
\]

\[
	x(2) \approx 1.33 \text{ m}
\]

Por lo tanto, la partícula estará aproximadamente en \( x(2) = 1.33 \) m cuando \( t = 2 \) segundos.


\end{document}