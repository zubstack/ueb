\documentclass[answers]{exam} % Clase para exámenes con respuestas
\usepackage[english,spanish]{babel} % Soporte para inglés y español
\usepackage[autostyle]{csquotes} % Manejo de citas
\usepackage{amsmath, amssymb} % Paquetes para matemáticas avanzadas
\usepackage{graphicx} % Inclusión de gráficos
\usepackage{enumitem} % Personalización de listas enumeradas
\usepackage[letterpaper,top=2cm,bottom=2cm,left=3cm,right=3cm,marginparwidth=1.75cm]{geometry} % Configuración de márgenes
\usepackage[colorlinks=true, allcolors=blue]{hyperref} % Enlaces con color
\usepackage{tikz}

\usepackage{array}   % for adjusting row height
\renewcommand{\arraystretch}{1.5} % adjust the vertical spacing between rows

\renewcommand{\solutiontitle}{\noindent\textbf{Respuesta:}\par\noindent} % Personalización del título de respuestas
\renewcommand{\familydefault}{\sfdefault}

% Configuración de encabezado y pie de página
\pagestyle{headandfoot}
\firstpageheader{Universidad de Bolívar}{}{Noviembre 12, 2024} 
\runningheader{Universidad de Bolívar}{}{Física}
\firstpagefooter{}{\thepage}{}
\runningfooter{}{\thepage}{}

\begin{document}

\begin{center}
    \large\textbf{Trabajo Autónomo 2.10 - Fundamentos de Física para Ingeniería}\\[1em]
    \large Segundo Ciclo \enquote*{A} - Ingeniería de Software\\[1em]
\end{center}

\vspace{0.5cm}
\noindent
\large\textbf{Tema:} CAPACITORES Y DIELÉCTRICOS \\
\large\textbf{Estudiante:} Ariel Alejandro Calderón
\vspace{0.5cm}

\begin{questions}

    % Question 1
    \question \large\textbf{Dado un capacitor de placas paralelas distanciadas 1 mm, Hallar el área de las placas para que su capacitancia sea de 1 F, 1 uF, 1 pF.:}
  
        Para calcular el área \( A \) de las placas de un capacitor de placas paralelas con una capacitancia \( C \), la fórmula es:
        \[
        C = \frac{\varepsilon_0 A}{d}
        \]

        donde:\\
        - \( C \) es la capacitancia deseada,\\
        - \( \varepsilon_0 = 8.85 \times 10^{-12} \, \text{F/m} \) es la permitividad del vacío,\\
        - \( d = 1 \, \text{mm} = 1 \times 10^{-3} \, \text{m} \) es la distancia entre las placas.\\

        
        Para hallar \( A \), despejamos en la ecuación:
        \[
        A = \frac{C \cdot d}{\varepsilon_0}
        \]
        
        1. Para \( C = 1 \, \text{F} \):
           \[
           A = \frac{1 \, \text{F} \times 1 \times 10^{-3} \, \text{m}}{8.85 \times 10^{-12} \, \text{F/m}} = 1.13 \times 10^8 \, \text{m}^2
           \]
        
        2. Para \( C = 1 \, \mu \text{F} = 1 \times 10^{-6} \, \text{F} \):
           \[
           A = \frac{1 \times 10^{-6} \, \text{F} \times 1 \times 10^{-3} \, \text{m}}{8.85 \times 10^{-12} \, \text{F/m}} = 113 \, \text{m}^2
           \]
        
        3. Para \( C = 1 \, \text{pF} = 1 \times 10^{-12} \, \text{F} \):
           \[
           A = \frac{1 \times 10^{-12} \, \text{F} \times 1 \times 10^{-3} \, \text{m}}{8.85 \times 10^{-12} \, \text{F/m}} = 0.113 \, \text{m}^2
           \]
        
    \begin{center}
        \begin{tikzpicture}
            % Configuración de estilos
            \tikzset{
                capacitor/.style={fill=blue!30, draw=black},
                field/.style={->, thick, color=red},
            }
        
            % Dimensiones de las placas
            \def\plateWidth{3}    % Ancho de las placas
            \def\plateHeight{0.3} % Altura de las placas
            \def\distance{1.5}    % Distancia entre las placas
        
            % Dibujar placa superior
            \fill[capacitor] (-\plateWidth/2, \distance/2) rectangle (\plateWidth/2, \distance/2 + \plateHeight);
            \node at (0, \distance/2 + \plateHeight + 0.3) {Placa positiva};
        
            % Dibujar placa inferior
            \fill[capacitor] (-\plateWidth/2, -\distance/2) rectangle (\plateWidth/2, -\distance/2 - \plateHeight);
            \node at (0, -\distance/2 - \plateHeight - 0.3) {Placa negativa};
        
            % Etiqueta de la distancia
            \draw[thick] (\plateWidth/2 + 0.2, -\distance/2) -- (\plateWidth/2 + 0.2, \distance/2);
            \node[right] at (\plateWidth/2 + 0.5, 0) {$d = 1 \, \text{mm}$};
        
         %    % Flechas del campo eléctrico
         %    \foreach \y in {-0.7, -0.3, 0, 0.3, 0.7} {
         %        \draw[field] (-\plateWidth/2 + 0.5, \y) -- (\plateWidth/2 - 0.5, \y)
         %    }
         %    \node[red] at (0, 1) {Campo eléctrico};
        
            % Etiqueta del área
            \draw[dashed] (-\plateWidth/2, -\distance/2 - 0.5) -- (\plateWidth/2, -\distance/2 - 0.5);
            \node[below] at (0, -\distance/2 - 0.7) {$A$ (Área de las placas)};
        
        \end{tikzpicture}
    \end{center} 
        
     \vspace{0.5cm}
    % Question 2
    \question \large\textbf{Calcular la capacitancia de un cable coaxial en el vacío cuyos radios son de 1 mm y 2 mm y longitud 2m.}
 
   
        Para calcular la capacitancia \( C \) de un cable coaxial en el vacío, se usa la fórmula:
        \[
        C = \frac{2 \pi \varepsilon_0 L}{\ln\left(\frac{b}{a}\right)}
        \]

        donde:\\
        
        - \( L = 2 \, \text{m} \) es la longitud del cable,\\
        - \( a = 1 \, \text{mm} = 1 \times 10^{-3} \, \text{m} \) es el radio interno,\\
        - \( b = 2 \, \text{mm} = 2 \times 10^{-3} \, \text{m} \) es el radio externo.\\
        
        Sustituyendo los valores:
        \[
        C = \frac{2 \pi \times 8.85 \times 10^{-12} \, \text{F/m} \times 2 \, \text{m}}{\ln\left(\frac{2 \times 10^{-3} \, \text{m}}{1 \times 10^{-3} \, \text{m}}\right)}
        \]
        
        Calculamos el valor del logaritmo natural:
        \[
        \ln\left(\frac{2 \times 10^{-3}}{1 \times 10^{-3}}\right) = \ln(2) \approx 0.693
        \]
        
        Sustituyendo en la ecuación:
        \[
        C = \frac{2 \pi \times 8.85 \times 10^{-12} \times 2}{0.693} \approx 1.60 \times 10^{-10} \, \text{F} = 160 \, \text{pF}
        \]
        
        Por lo tanto, la capacitancia del cable coaxial es aproximadamente \( 160 \, \text{pF} \).
        
     \vspace{0.5cm}
    % Question 3
    \question \large\textbf{Calcular la capacitancia de un capacitor esférico de radios 2 mm y 4 mm
    respectivamente si el espacio entre los conductores está relleno de aceite, cuya
    permitividad eléctrica relativa es 4.}

    
        Para calcular la capacitancia \( C \) de un capacitor esférico con un dieléctrico de permitividad relativa \( \varepsilon_r = 4 \), usamos la fórmula:
        \[
        C = 4 \pi \varepsilon_0 \varepsilon_r \frac{a \cdot b}{b - a}
        \]

        donde:\\
        - \( \varepsilon_r = 4 \) es la permitividad relativa del dieléctrico,\\
        - \( a = 2 \, \text{mm} = 2 \times 10^{-3} \, \text{m} \) es el radio interno,\\
        - \( b = 4 \, \text{mm} = 4 \times 10^{-3} \, \text{m} \) es el radio externo.\\
        
        Sustituyendo los valores en la fórmula:
        \[
        C = 4 \pi \times 8.85 \times 10^{-12} \, \text{F/m} \times 4 \times \frac{2 \times 10^{-3} \times 4 \times 10^{-3}}{4 \times 10^{-3} - 2 \times 10^{-3}}
        \]
        
        Calculamos el valor del denominador:
        \[
        4 \times 10^{-3} - 2 \times 10^{-3} = 2 \times 10^{-3} \, \text{m}
        \]
        
        Sustituyendo en la ecuación:
        \[
        C = 4 \pi \times 8.85 \times 10^{-12} \times 4 \times \frac{8 \times 10^{-6}}{2 \times 10^{-3}}
        \]
        
        Simplificando:
        \[
        C = 4 \pi \times 8.85 \times 10^{-12} \times 4 \times 4 \times 10^{-3} = 1.78 \times 10^{-11} \, \text{F} = 17.8 \, \text{pF}
        \]
        
        Por lo tanto, la capacitancia del capacitor esférico es aproximadamente \( 17.8 \, \text{pF} \).
      
        
     \vspace{0.5cm}
    % Question 4
    \question \large\textbf{Calcular la energía almacenada por los condensadores de los ejercicios anteriores si están sometidos a una diferencia de potencial de 100 voltios.}

  
        La energía almacenada \( U \) en un capacitor se calcula con la fórmula:
        \[
        U = \frac{1}{2} C V^2
        \]
        donde:\\
        - \( C \) es la capacitancia del capacitor,\\
        - \( V = 100 \, \text{V} \) es la diferencia de potencial aplicada.\\
        
        Calculamos la energía almacenada para cada uno de los capacitores obtenidos en los ejercicios anteriores.
        
        1. Capacitor de placas paralelas (Ejercicio 1):
           - Para \( C = 1 \, \text{F} \):
             \[
             U = \frac{1}{2} \times 1 \, \text{F} \times (100 \, \text{V})^2 = \frac{1}{2} \times 1 \times 10000 = 5000 \, \text{J}
             \]
           - Para \( C = 1 \, \mu \text{F} = 1 \times 10^{-6} \, \text{F} \):
             \[
             U = \frac{1}{2} \times 1 \times 10^{-6} \times 10000 = 0.005 \, \text{J}
             \]
           - Para \( C = 1 \, \text{pF} = 1 \times 10^{-12} \, \text{F} \):
             \[
             U = \frac{1}{2} \times 1 \times 10^{-12} \times 10000 = 5 \times 10^{-9} \, \text{J} = 5 \, \text{nJ}
             \]
        
        2. Capacitor coaxial (Ejercicio 2):
           - Para \( C \approx 160 \, \text{pF} = 160 \times 10^{-12} \, \text{F} \):
             \[
             U = \frac{1}{2} \times 160 \times 10^{-12} \times (100)^2 = 8 \times 10^{-7} \, \text{J} = 0.8 \, \mu \text{J}
             \]
        
        3. Capacitor esférico (Ejercicio 3):
           - Para \( C \approx 17.8 \, \text{pF} = 17.8 \times 10^{-12} \, \text{F} \):
             \[
             U = \frac{1}{2} \times 17.8 \times 10^{-12} \times (100)^2 = 8.9 \times 10^{-8} \, \text{J} = 0.089 \, \mu \text{J}
             \]
        

        

     \vspace{0.5cm}
    % Question 5
    \question \large\textbf{Resolver los cuatro ejercicios anteriores, si los condensadores están rellenos de un dieléctrico cuya permitividad relativa es de 52, y una diferencia de potencial de 50
    voltios.}
    
        Para resolver los ejercicios anteriores con un dieléctrico cuya permitividad relativa es \( \varepsilon_r = 52 \) y una diferencia de potencial de \( V = 50 \, \text{V} \), recalculamos la capacitancia de cada capacitor y la energía almacenada en cada caso.
        
        La capacitancia con un dieléctrico se obtiene multiplicando la capacitancia en el vacío \( C_0 \) por \( \varepsilon_r \):
        \[
        C = \varepsilon_r \cdot C_0
        \]
        
        1. Capacitor de placas paralelas (Ejercicio 1):\\
        - Para \( C_0 = 1 \, \text{F} \):
          \[
          C = 52 \times 1 \, \text{F} = 52 \, \text{F}
          \]
        - Para \( C_0 = 1 \, \mu \text{F} = 1 \times 10^{-6} \, \text{F} \):
          \[
          C = 52 \times 1 \times 10^{-6} \, \text{F} = 52 \times 10^{-6} \, \text{F} = 52 \, \mu \text{F}
          \]
        - Para \( C_0 = 1 \, \text{pF} = 1 \times 10^{-12} \, \text{F} \):
          \[
          C = 52 \times 1 \times 10^{-12} \, \text{F} = 52 \, \text{pF}
          \]
        
        Calculamos la energía almacenada con \( V = 50 \, \text{V} \):
        \[
        U = \frac{1}{2} C V^2
        \]
        
        - Para \( C = 52 \, \text{F} \):
          \[
          U = \frac{1}{2} \times 52 \times (50)^2 = \frac{1}{2} \times 52 \times 2500 = 65000 \, \text{J}
          \]
        - Para \( C = 52 \, \mu \text{F} \):
          \[
          U = \frac{1}{2} \times 52 \times 10^{-6} \times 2500 = 0.065 \, \text{J}
          \]
        - Para \( C = 52 \, \text{pF} \):
          \[
          U = \frac{1}{2} \times 52 \times 10^{-12} \times 2500 = 6.5 \times 10^{-8} \, \text{J} = 65 \, \text{nJ}
          \]
        
        2. Capacitor coaxial (Ejercicio 2):\\
        Para \( C_0 \approx 160 \, \text{pF} \):
        \[
        C = 52 \times 160 \, \text{pF} = 8320 \, \text{pF}
        \]
        
        La energía almacenada es:
        \[
        U = \frac{1}{2} \times 8320 \times 10^{-12} \times (50)^2 = 1.04 \times 10^{-6} \, \text{J} = 1.04 \, \mu \text{J}
        \]
        
        3. Capacitor esférico (Ejercicio 3):\\
        Para \( C_0 \approx 17.8 \, \text{pF} \):
        \[
        C = 52 \times 17.8 \, \text{pF} = 925.6 \, \text{pF}
        \]
        
        La energía almacenada es:
        \[
        U = \frac{1}{2} \times 925.6 \times 10^{-12} \times (50)^2 = 1.16 \times 10^{-7} \, \text{J} = 0.116 \, \mu \text{J}
        \]
        
   
       
        
     \vspace{0.5cm}
    % Question 6
    \question \large\textbf{En las circunstancias del ejercicio 1, ¿qué cantidad de carga almacenarán cada condensador?}
   
        La cantidad de carga \( Q \) almacenada en un capacitor se calcula con la fórmula:
        \[
        Q = C \cdot V
        \]
        donde:\\
        - \( C \) es la capacitancia del capacitor,\\
        - \( V \) es la diferencia de potencial aplicada.\\
        
        Utilizando los valores de capacitancia para cada caso en el Ejercicio 1 y una diferencia de potencial de \( V = 100 \, \text{V} \), calculamos la carga almacenada en cada capacitor.
        
        Capacitor de placas paralelas (Ejercicio 1):\\
        1. Para \( C = 1 \, \text{F} \):
           \[
           Q = 1 \, \text{F} \times 100 \, \text{V} = 100 \, \text{C}
           \]
        2. Para \( C = 1 \, \mu \text{F} = 1 \times 10^{-6} \, \text{F} \):
           \[
           Q = 1 \times 10^{-6} \, \text{F} \times 100 \, \text{V} = 1 \times 10^{-4} \, \text{C} = 100 \, \mu \text{C}
           \]
        3. Para \( C = 1 \, \text{pF} = 1 \times 10^{-12} \, \text{F} \):
           \[
           Q = 1 \times 10^{-12} \, \text{F} \times 100 \, \text{V} = 1 \times 10^{-10} \, \text{C} = 0.1 \, \text{nC}
           \]
        
        Capacitor coaxial (Ejercicio 2):\\
        Para \( C \approx 160 \, \text{pF} = 160 \times 10^{-12} \, \text{F} \):
        \[
        Q = 160 \times 10^{-12} \, \text{F} \times 100 \, \text{V} = 1.6 \times 10^{-8} \, \text{C} = 16 \, \text{nC}
        \]
        
        Capacitor esférico (Ejercicio 3):\\
        Para \( C \approx 17.8 \, \text{pF} = 17.8 \times 10^{-12} \, \text{F} \):
        \[
        Q = 17.8 \times 10^{-12} \, \text{F} \times 100 \, \text{V} = 1.78 \times 10^{-9} \, \text{C} = 1.78 \, \text{nC}
        \]
        
      
        

\end{questions}

\end{document}