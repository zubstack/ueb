\documentclass[answers]{exam} % Clase para exámenes con respuestas
\usepackage[english,spanish]{babel} % Soporte para inglés y español
\usepackage[autostyle]{csquotes} % Manejo de citas
\usepackage{amsmath, amssymb} % Paquetes para matemáticas avanzadas
\usepackage{graphicx} % Inclusión de gráficos
\usepackage{enumitem} % Personalización de listas enumeradas
\usepackage[letterpaper,top=2cm,bottom=2cm,left=3cm,right=3cm,marginparwidth=1.75cm]{geometry} % Configuración de márgenes
\usepackage[colorlinks=true, allcolors=blue]{hyperref} % Enlaces con color
\usepackage{circuitikz}

\usepackage{array}   % for adjusting row height
\renewcommand{\arraystretch}{1.5} % adjust the vertical spacing between rows

\renewcommand{\solutiontitle}{\noindent\textbf{Respuesta:}\par\noindent} % Personalización del título de respuestas
\renewcommand{\sfdefault}{\familydefault}

% Configuración de encabezado y pie de página
\pagestyle{headandfoot}
\firstpageheader{Universidad de Bolívar}{}{Diciembre 6, 2024} 
\runningheader{Universidad de Bolívar}{}{Física}
\firstpagefooter{}{\thepage}{}
\runningfooter{}{\thepage}{}

\begin{document}

\begin{center}
	\large\textbf{Trabajo Autónomo 2.14 - Fundamentos de Física para Ingeniería}\\[1em]
	\large Segundo Ciclo \enquote*{A} - Ingeniería de Software\\[1em]
\end{center}

\vspace{0.5cm}
\noindent
\large\textbf{Tema:} Circuitos resistivos\\
\large\textbf{Estudiante:} Ariel Alejandro Calderón
\vspace{0.5cm}

\begin{questions}
	\question \large\textbf{En los siguientes circuitos eléctricos encontrar: }

	\begin{parts}
		\part La corriente eléctrica en todas las ramas del circuito.
		\part La diferencia de potencial entre los puntos a y b.
		\part La potencia que disipa la resistencia $R_3$.
		\part La potencia que entrega cada fuente.
	\end{parts}

	\section*{Circuito 1}

	\begin{center}

		\begin{circuitikz}

			\draw
			(0,0) to[V, v={$110V$}] (0,4)  % 
			to[R, l_={$R_1 = 10 \ \Omega$}, v^={}] (4,4)
			to[R, l_={$R_2 = 15 \ \Omega$}, v^={}] (4,0)
			to[R, l_={$R_3 = 10 \ \Omega$}, v^={}] (0,0)

			% Malla 2
			(4,4) to[R, l_={$R_4 = 15 \ \Omega$}, v^={}] (8,4)
			to[R, l_={$R_5 = 5 \ \Omega$}, v^={}] (8,0)
			to[R, l_={$R_6 = 20 \ \Omega$}, v^={}] (4,0)

			;
		\end{circuitikz}
	\end{center}



	\textbf{Análisis de Mallas}

	\[
		\text{Malla 1:} \quad 110 - 10 I_1 - 10 (I_1 - I_2) - 15 I_1 = 0
	\]
	\[
		110 - 10 I_1 - 10 I_1 + 10 I_2 - 15 I_1 = 0
	\]
	\[
		- 35 I_1 + 10 I_2 = - 110
	\]

	\[
		\text{Malla 2:} \quad -10 (I_1 - I_2) - 15 I_2 - 5 I_2 - 20 I_2 = 0
	\]
	\[
		-10 I_1 + 10 I_2 - 15 I_2 - 5 I_2 - 20 I_2 = 0
	\]
	\[
		-10 I_1 - 50 I_2 = 0 \implies I_1 = 5 I_2
	\]

	\[
		\text{Reemplazo:} \quad - 35 ( 5 I_2 )+ 10 I_2 = - 110
	\]
	\[
		I_2 = \dfrac{-110}{-165} \implies \boxed{I_2 = \dfrac{2}{3} \text{ A} ; I_1 = \dfrac{10}{3} \text{ A}}
	\]

	\newpage
	\begin{parts}
		\part \textbf{Corrientes de rama}
		\begin{enumerate}[label=•]
			\item $ i_1 = I_1 = \dfrac{2}{3} \text{ A}$
			\item $ i_2 = I_1 - I_2 =  \dfrac{8}{3} \text{ A}$
			\item $ i_3 = I_2 = \dfrac{10}{3} \text{ A}$
		\end{enumerate}
		\part \textbf{Diferencia de potencia entre a y b}
		\[
			V_{ab}= V_{R_3} = i_2 * R_3 = \dfrac{80}{3} \text{ V}
		\]
		\part \textbf{Potencia disipada por $R_3$}

		\[
			\boxed{P = I^2 R }
		\]
		\[
			P_{R_3} = (i_2)^2 \times R_3 = 71.10 \text{ W}
		\]



		\part \textbf{Potencia que entrega cada fuente}

		\[
			\boxed{P=I*R}
		\]

		\[
			P_1 = i_1 * V_1 = 366.6\text{ W}
		\]

	\end{parts}

	\vspace{0.5cm}
	\section*{Circuito 2}

	\begin{center}

		\begin{circuitikz}

			\draw
			(0,0) to[V, v={$V_1 = 100V$}] (0,4)  % 
			to[R, l_={$R_1 = 10 \ \Omega$}, v^={}] (4,4)
			to[R, l_={$R_2 = 10 \ \Omega$}, v^={}] (4,0)
			to[R, l_={$R_4 = 10 \ \Omega$}, v^={}] (0,0)

			% Malla 2
			(4,4) to[R, l_={$R_2 = 10 \ \Omega$}, v^={}] (8,4)
			to[V, v={$V_2 = 60 V$}] (8,0)
			to[R, l_={$R_5 = 10 \ \Omega$}, v^={}] (4,0)
			;

		\end{circuitikz}
	\end{center}



	\textbf{Análisis de Mallas}

	\[
		\text{Malla 1:} \quad 110 - 10 I_1 - 10 (I_1 - I_2) - 10 I_1 = 0
	\]
	\[
		110 - 10 I_1 - 10 I_1 + 10 I_2 - 10 I_1 = 0
	\]
	\[
		- 30 I_1 + 10 I_2 = - 110
	\]

	\[
		\text{Malla 2:} \quad -10 (I_2 - I_1) - 10 I_2 - 10 I_2 - 10 I_2 + 60= 0
	\]
	\[
		10 I_1  - 10 I_2 - 10 I_2 - 10 I_2 = -60
	\]
	\[
		10 I_1 - 30 I_2 = -60 \implies I_1 = 3I_2 - 6
	\]
	\[
		-30(3I_2-6)+10I_2 = -100 \implies -90I_2 +180 + 10I_2=-100
	\]
	\[
		-80I_2 = -280 \implies \boxed{I_2= 3.5 \text{ A}; I_1 = 4.5 \text{ A}}
	\]


	\begin{parts}
		\part \textbf{Corrientes de rama}
		\begin{enumerate}[label=•]
			\item $ i_1 = I_1 = 4.5 \text{ A}$
			\item $ i_2 = I_1 - I_2 =  1\text{ A}$
			\item $ i_3 = I_2 = 3.5 \text{ A}$
		\end{enumerate}
		\part \textbf{Diferencia de potencia entre a y b}
		\[
			V_{ab}= V_{R_3} = i_2 * R_3 = 10 \text{ V}
		\]
		\part \textbf{Potencia disipada por $R_3$}

		\[
			\boxed{P = I^2 R }
		\]
		\[
			P_{R_3} = (i_2)^2 \times R_3 = 10 \text{ W}
		\]



		\part \textbf{Potencia que entrega cada fuente}

		\[
			\boxed{P=I*R}
		\]

		\[
			P_1 = i_1 * V_1 = 450\text{ W}
		\]
		\[
			P_2 = i_3 * V_2 = 210\text{ W}
		\]

	\end{parts}

	\vspace{0.5cm}
	\section*{Circuito 3}

	\begin{center}

		\begin{circuitikz}

			\draw
			(0,0) to[V, v={$V_1 = 100V$}] (0,5)  % 
			to[R, l_={$R_1 = 20 \ \Omega$}, v^={}] (5,5)
			(5,2.4) to[V, v={$V_2 = 120 V$}] (5,5)
			(5,2.4) to[R, l_={$R_3 = 10 \ \Omega$}, v^={}] (5,0)
			to[R, l_={$R_4 = 10 \ \Omega$}, v^={}] (0,0)

			% Malla 2
			(5,5) to[R, l_={$R_2 = 10 \ \Omega$}, v^={}] (10,5)
			(10,0)to[V, v={$V_3 = 100 V$}] (10,5)
			(10,0)to[R, l_={$R_5 = 20 \ \Omega$}, v^={}] (5,0)
			;

		\end{circuitikz}
	\end{center}



	\textbf{Análisis de Mallas}

	\[
		\text{Malla 1:} \quad 100 - 20 I_1 - 10 I_1 - 10 (I_1 - I_2) - 120 = 0
	\]
	\[
		-40 I_1 + 10 I_2 = 20 
	\]

	\[
		\text{Malla 2:} \quad -120 + 10 I_2 + 10 I_2 + 20 (I_2 - I_1) - 100 = 0
	\]
	\[
		-20 I_1 + 40 I_2 = 220 
	\]
	\newpage

	\[
		-80 I_1 + 20 I_2 = 40 
	\]

	\[
		(-80 I_1 + 20 I_2) - (-20 I_1 + 40 I_2) = 40 - 220
	\]
	\[
		-60 I_1 - 20 I_2 = -180
	\]
	\[
		3 I_1 + I_2 = 9 
	\]

	\[
		-40 I_1 + 10 (9 - 3 I_1) = 20
	\]
	\[
		\boxed{I_1 = 1 \, \text{A}, \quad I_2 = 6 \, \text{A}}
	\]

	\begin{parts}
		\part \textbf{Corrientes de rama}
		\begin{enumerate}[label=•]
			\item $ i_1 = I_1 =  1 \, \text{A}$
			\item $ i_2 = I_1 - I_2 = -5 \text{ A}$
			\item $ i_3 = I_2 = 6 \text{ A}$
		\end{enumerate}
		\part \textbf{Diferencia de potencia entre a y b}
		\[
			V_{ab}= V_2 + V_{R_3} =V_2 + i_2 * R_3 = 70 \text{ V}
		\]
		\part \textbf{Potencia disipada por $R_3$}

		\[
			\boxed{P = I^2 R }
		\]
		\[
			P_{R_3} = (i_2)^2 \times R_3 = 250 \text{ W}
		\]



		\part \textbf{Potencia que entrega cada fuente}

		\[
			\boxed{P=I*R}
		\]

		\[
			P_1 = i_1 * V_1 = 100\text{ W}
		\]
		\[
			P_2 = |i_2| * V_2 = 600\text{ W}
		\]
		\[
			P_3 = i_3 * V_3 = 600\text{ W}
		\]

	\end{parts}

	\vspace{0.5cm}


\end{questions}

\end{document}



% \section*{Cálculo de la potencia disipada por cada resistencia}

% La potencia disipada en cada resistencia se calcula con la fórmula \( P = I^2 R \).



% \section*{Verificación de la conservación de energía}

% La potencia total generada por las fuentes es:

% \[
% P_{\text{total}} = 100 \times 1 + 120 \times 5 + 100 \times 6 = 1300 \, \text{W}
% \]

% La potencia total disipada es:

% \[
% P_{\text{disipada}} = 20 + 360 + 490 + 250 + 500 = 1620 \, \text{W}
% \]