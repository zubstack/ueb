\documentclass{article}
\usepackage{amsmath}

\begin{document}

\section*{Aproximaciones usando las Sumas de Riemann}

Sea \( f(x) \) una función continua en el intervalo \([0, \frac{5}{2}]\), queremos aproximar el área bajo la curva en este intervalo mediante las Sumas de Riemann. Dividimos el intervalo en 10 subintervalos y calculamos la suma izquierda, derecha y del punto medio.

Dividimos el intervalo en 10 subintervalos de igual tamaño, con \(\Delta x = \frac{1}{4}\).

Los intervalos son:
\[
    \left[0, \frac{1}{4}\right], \quad \left[\frac{1}{4}, \frac{1}{2}\right], \quad \left[\frac{1}{2}, \frac{3}{4}\right], \quad \left[\frac{3}{4}, 1\right],\quad \left[1, \frac{5}{4}\right], 
\]
\[
    \left[\frac{5}{4}, \frac{3}{2}\right],\quad \left[\frac{3}{2}, \frac{7}{4}\right],\quad \left[\frac{7}{4}, 2\right],\quad \left[2, \frac{9}{4}\right], \quad \left[\frac{9}{4}, \frac{5}{2}\right]
\]

\section{Suma Izquierda}
La suma de Riemann izquierda se define como:
\[
L = \sum_{i=1}^{10} f(x_{i-1}) \cdot \Delta x
\]
donde los valores de \( x_{i-1} \) son los puntos izquierdos de cada intervalo.

Por lo tanto,
\[
L = f(0) \cdot \frac{1}{4} + f\left(\frac{1}{4}\right) \cdot \frac{1}{4} + f\left(\frac{1}{2}\right) \cdot \frac{1}{4} + \dots 
\]

\section{Suma Derecha}
La suma de Riemann derecha se define como:
\[
R = \sum_{i=1}^{10} f(x_{i}) \cdot \Delta x
\]
donde los valores de \( x_{i} \) son los puntos derechos de cada intervalo.

Por lo tanto,
\[
R = f\left(\frac{1}{4}\right) \cdot \frac{1}{4} + f\left(\frac{1}{2}\right) \cdot \frac{1}{4} + \dots 
\]
\section{Suma del Punto Medio}
La suma de Riemann del punto medio se define como:
\[
M = \sum_{i=1}^{10} f\left(\frac{x_{i-1} + x_i}{2}\right) \cdot \Delta x
\]
donde los valores de \( \frac{x_{i-1} + x_i}{2} \) son los puntos medios de cada intervalo:
\[
\frac{0 + \frac{1}{4}}{2} = \frac{1}{8}, \quad \frac{\frac{1}{4} + \frac{1}{2}}{2} = \frac{3}{8}, \quad \dots
\]
Por lo tanto,
\[
M = f\left(\frac{1}{8}\right) \cdot \frac{1}{4} + f\left(\frac{3}{8}\right) \cdot \frac{1}{4} + f\left(\frac{5}{8}\right) \cdot \frac{1}{4} + \dots
\]

\end{document}
