\documentclass{article}
\usepackage[utf8]{inputenc}
\usepackage{amssymb}
\usepackage{amsmath}

\begin{document}

\section*{Solución del Examen de Matemáticas}

\textbf{Pregunta 1}

Formaliza las siguientes proposiciones:

\textbf{a)}

\textit{“Si tuvieran que justificarse ciertos hechos por su enorme tradición entonces, si estos hechos son inofensivos y respetan a todo ser viviente y al medio ambiente, no habría ningún problema. Pero si los hechos son bárbaros o no respetuosos con los seres vivientes o el medio ambiente, entonces habría que dejar de justificarlos o no podríamos considerarnos dignos de nuestro tiempo.”}

Para formalizar esta proposición, usaremos las siguientes variables:

- \( J \): "Los hechos se justifican por su enorme tradición"
- \( I \): "Los hechos son inofensivos"
- \( R \): "Los hechos respetan a todo ser viviente y al medio ambiente"
- \( P \): "No hay ningún problema"
- \( B \): "Los hechos son bárbaros"
- \( N \): "Los hechos no respetan a los seres vivientes o al medio ambiente"
- \( D \): "Habría que dejar de justificarlos"
- \( T \): "No podemos considerarnos dignos de nuestro tiempo"

La proposición puede expresarse como:

\[ J \rightarrow [(I \land R) \rightarrow P] \land [(B \lor N) \rightarrow (D \lor T)] \]

\
\section*{Pregunta 2}

\textit{Todo número entero o es primo o es compuesto. Si es compuesto, es un producto de factores primos, y si es un producto de factores primos, entonces es divisible por ellos. Pero si un número entero es primo, no es compuesto, aunque es divisible por sí mismo y por la unidad, y consiguientemente, también divisible por números primos. Por tanto, todo número entero es divisible por números primos.}

Para formalizar esta proposición, usaremos las siguientes variables:

- \( E(x) \): "x es un número entero"
- \( P(x) \): "x es primo"
- \( C(x) \): "x es compuesto"
- \( F(x) \): "x es un producto de factores primos"
- \( D(x, y) \): "x es divisible por y"

La proposición puede expresarse como:

\[ \forall x (E(x) \rightarrow [P(x) \lor C(x)]) \]
\[ \forall x (C(x) \rightarrow F(x)) \]
\[ \forall x (F(x) \rightarrow \forall y (D(x, y) \rightarrow P(y))) \]
\[ \forall x (P(x) \rightarrow \neg C(x)) \]
\[ \forall x (P(x) \rightarrow (D(x, x) \land D(x, 1))) \]
\[ \forall x (P(x) \rightarrow \forall y (D(x, y) \rightarrow P(y))) \]

\section*{Pregunta 3}

La cantidad de correos electrónicos no leídos en la cuenta de Sylvia es 75. Esta cantidad aumenta en 10 correos electrónicos no leídos por día. La función \( N(t) = 75 + 10t \) representa la relación entre el número de correos electrónicos, \( N \), y el tiempo, \( t \), medido en días.

\textbf{a.} Determinar la variable independiente y dependiente.

Para la función \( N(t) = 75 + 10t \):

- La variable independiente es \( t \) (el tiempo en días).
- La variable dependiente es \( N \) (el número de correos electrónicos no leídos).

\textbf{b.} Encuentre \( N(5) \). Explique qué significa este resultado.

Para encontrar \( N(5) \), sustituimos \( t = 5 \) en la función:

\[ N(5) = 75 + 10 \cdot 5 = 75 + 50 = 125 \]

Este resultado significa que después de 5 días, Sylvia tendrá 125 correos electrónicos no leídos en su cuenta.

\section*{Pregunta 4}

El costo diario para la imprenta por imprimir un libro se realiza mediante la función \( C(x) = 3.25x + 1500 \) donde \( C \) es el costo diario total y \( x \) es el número de libros impresos.

\textbf{a.} Determine la variable independiente y dependiente.

Para la función \( C(x) = 3.25x + 1500 \):

- La variable independiente es \( x \) (el número de libros impresos).
- La variable dependiente es \( C \) (el costo diario total).

\textbf{b.} Encuentre \( C(0) \). Explique qué significa este resultado.

Para encontrar \( C(0) \), sustituimos \( x = 0 \) en la función:

\[ C(0) = 3.25 \cdot 0 + 1500 = 1500 \]

Este resultado significa que el costo diario total cuando no se imprime ningún libro es \$1500, que probablemente representa los costos fijos de la imprenta.

\textbf{c.} Encuentre \( C(1000) \). Explique qué significa este resultado.

Para encontrar \( C(1000) \), sustituimos \( x = 1000 \) en la función:

\[ C(1000) = 3.25 \cdot 1000 + 1500 = 3250 + 1500 = 4750 \]

Este resultado significa que el costo diario total para imprimir 1000 libros es \$4750.

\newpage

\section*{Pregunta 5}

Si 10 personas se dan la mano cada una, ¿cuántos apretones de manos se produjeron? ¿Qué tiene que ver esta pregunta con la teoría de grafos?

Para determinar el número de apretones de manos, podemos usar la fórmula combinatoria para elegir 2 personas de un grupo de 10, que es \(\binom{10}{2}\):

\[ \binom{10}{2} = \frac{10!}{2!(10-2)!} = \frac{10 \cdot 9}{2 \cdot 1} = 45 \]

Se produjeron 45 apretones de manos.

En la teoría de grafos, esto representa un grafo completo \( K_{10} \)

\section*{Pregunta 6}

Determina por extensión el siguiente conjunto:
\[ A = \left\{ x^2 + 1 \mid x \in \mathbb{Z} \land -3 < x \leq 4 \right\} \]

Primero, encontramos los valores de \( x \) en el intervalo dado:
\[ x = -2, -1, 0, 1, 2, 3, 4 \]

Ahora, calculamos \( x^2 + 1 \) para cada valor de \( x \):
\[ 
\begin{align*}
(-2)^2 + 1 & = 4 + 1 = 5 \\
(-1)^2 + 1 & = 1 + 1 = 2 \\
0^2 + 1 & = 0 + 1 = 1 \\
1^2 + 1 & = 1 + 1 = 2 \\
2^2 + 1 & = 4 + 1 = 5 \\
3^2 + 1 & = 9 + 1 = 10 \\
4^2 + 1 & = 16 + 1 = 17 \\
\end{align*}
\]

Por lo tanto, el conjunto \( A \) es:
\[ A = \{1, 2, 5, 10, 17\} \]

La suma de los elementos del conjunto \( A \) es:
\[ 1 + 2 + 5 + 10 + 17 = 35 \]

\section*{Pregunta 7}

Encuentra el dominio de la siguiente relación \( R = \{(x, y) \mid y = x + 2\} \), dado los conjuntos \( A = \{1, 2, 3, 4, 6, 7\} \) y \( B = \{1, 2, 3, 4, 5, 6\} \).

Para que \( y \in B \), \( y \) debe satisfacer \( y = x + 2 \) y pertenecer al conjunto \( B \). Entonces:
\[ x + 2 \in B \]
\[ x \in A \]

Analizamos los valores de \( x \) de \( A \) y verificamos si \( x + 2 \in B \):

\[ 
\begin{align*}
1 + 2 & = 3 \in B \\
2 + 2 & = 4 \in B \\
3 + 2 & = 5 \in B \\
4 + 2 & = 6 \in B \\
6 + 2 & = 8 \notin B \\
7 + 2 & = 9 \notin B \\
\end{align*}
\]

Por lo tanto, el dominio de \( R \) es:
\[ \{1, 2, 3, 4\} \]

\section*{Pregunta 8}

Una persona tiene cuatro llaves y solo una llave cabe en la cerradura de una puerta. ¿Cuál es la probabilidad de que la puerta se pueda desbloquear en como máximo tres intentos?

La probabilidad de éxito en el primer intento es \( \frac{1}{4} \).

Si falla el primer intento, le quedan 3 llaves, así que la probabilidad de éxito en el segundo intento es \( \frac{1}{3} \).

Si falla el segundo intento, le quedan 2 llaves, así que la probabilidad de éxito en el tercer intento es \( \frac{1}{2} \).

La probabilidad de desbloquear la puerta en como máximo tres intentos es la suma de las probabilidades de éxito en cada intento:

\[
P(\text{éxito en 1er intento}) + P(\text{fracaso en 1er intento} \land \text{éxito en 2do intento}) + P(\text{fracaso en 1er y 2do intento} \land \text{éxito en 3er intento})
\]

\[
\begin{align*}
P(\text{éxito en 1er intento}) & = \frac{1}{4} \\
P(\text{fracaso en 1er intento} \land \text{éxito en 2do intento}) & = \frac{3}{4} \cdot \frac{1}{3} = \frac{1}{4} \\
P(\text{fracaso en 1er y 2do intento} \land \text{éxito en 3er intento}) & = \frac{3}{4} \cdot \frac{2}{3} \cdot \frac{1}{2} = \frac{1}{4} \\
\end{align*}
\]

\[
\frac{1}{4} + \frac{1}{4} + \frac{1}{4} = \frac{3}{4}
\]

Por lo tanto, la probabilidad de que la puerta se pueda desbloquear en como máximo tres intentos es \( \frac{3}{4} \) o 75\%.

\section*{Pregunta 9}

En una caja vacía ponemos 8 bolas azules, 4 bolas naranjas y 2 bolas verdes. Si sacamos primero una bola y después otra bola sin volver a poner la primera bola extraída dentro de la caja, ¿cuál es la probabilidad de que la primera bola sea azul y la segunda bola sea naranja?

Para encontrar esta probabilidad, primero calculamos el número total de bolas en la caja:

\[ 8 + 4 + 2 = 14 \]

La probabilidad de que la primera bola sea azul es:

\[ \frac{8}{14} \]

Después de sacar una bola azul, quedan 13 bolas en la caja, de las cuales 4 son naranjas. La probabilidad de que la segunda bola sea naranja, dado que la primera fue azul, es:

\[ \frac{4}{13} \]

Por lo tanto, la probabilidad de que la primera bola sea azul y la segunda sea naranja es el producto de estas dos probabilidades:

\[
\frac{8}{14} \times \frac{4}{13} = \frac{8 \cdot 4}{14 \cdot 13} = \frac{32}{182} = \frac{16}{91}
\]

Así, la probabilidad de que la primera bola sea azul y la segunda bola sea naranja es \(\frac{16}{91}\).


\end{document}
