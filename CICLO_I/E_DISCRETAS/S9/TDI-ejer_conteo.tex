\documentclass[answers]{exam} % Clase para exámenes con respuestas
\usepackage[english,spanish]{babel} % Soporte para inglés y español
\usepackage[autostyle]{csquotes} % Manejo de citas
\usepackage{amsmath, amssymb} % Paquetes para matemáticas avanzadas
\usepackage{graphicx} % Inclusión de gráficos
\usepackage{enumitem} % Personalización de listas enumeradas
\usepackage[letterpaper,top=2cm,bottom=2cm,left=3cm,right=3cm,marginparwidth=1.75cm]{geometry} % Configuración de márgenes
\usepackage[colorlinks=true, allcolors=blue]{hyperref} % Enlaces con color

\renewcommand{\solutiontitle}{\noindent\textbf{Respuesta:}\par\noindent} % Personalización del título de respuestas
\renewcommand{\familydefault}{\sfdefault}

% Configuración de encabezado y pie de página
\pagestyle{headandfoot}
\firstpageheader{Universidad de Bolívar}{}{\selectlanguage{spanish}\today} 
\runningheader{Universidad de Bolívar}{}{Cálculo I}
\firstpagefooter{}{\thepage}{}
\runningfooter{}{\thepage}{}

\begin{document}

\begin{center}
    \large\textbf{Trabajo Autónomo - Estructuras Discretas}\\[1em]
    \large Primer Ciclo \enquote*{A} - Ingeniería de Software\\[1em]
\end{center}

\vspace{0.5cm}
\large\textbf{Estudiante:} Ariel Alejandro Calderón
\vspace{0.5cm}

\begin{questions}

    % Question 1
    \question \large\textbf{¿Cuántas palabras de tres letras se pueden formar con cinco consonantes y tres vocales de modo que cada palabra comience y termine en consonante?}
    \begin{solution}
        \begin{itemize}
            \item Número de formas de elegir la primera consonante: 5
            \item Número de formas de elegir la vocal: 3
            \item Número de formas de elegir la última consonante: 5
        \end{itemize}


        Entonces, el número total de palabras posibles es:
        \[
            5 \times 3 \times 5 = 75
        \]
    \end{solution}

    \vspace{0.5cm}

    % Question 2
    \question \large\textbf{Determine el número de enteros de seis dígitos (que no comiencen con cero) en los que}
    \begin{enumerate}[label=\alph*.]
        \item Ningún dígito se pueda repetir.
        \item Se pueden repetir los dígitos.
    \end{enumerate}
    \begin{solution}
        \begin{enumerate}[label=\alph*.]
            \item Para números de seis dígitos donde ningún dígito se repita y no comiencen con cero:
                  \[
                      9 \times 9 \times 8 \times 7 \times 6 \times 5 = 136080
                  \]

            \item Para números de seis dígitos donde se puedan repetir los dígitos y no comiencen con cero:
                  \[
                      9 \times 10 \times 10 \times 10 \times 10 \times 10 = 900000
                  \]
        \end{enumerate}
    \end{solution}

    \vspace{0.5cm}

    % Question 3
    \question \large\textbf{¿Cuántas permutaciones existen para las ocho letras a,b,c,d,e,f,g,h?}
    \begin{solution}
        El número total de permutaciones de 8 letras distintas es:
        \[
            8! = 40320
        \]
    \end{solution}

    \vspace{0.5cm}

    % Question 4
    \question \large\textbf{¿De cuántas formas es posible ordenar los símbolos a,b,c,d,e,e,e,e,e de modo que ninguna e quede junto a otra?}
    \begin{solution}
        Primero, ordenamos las letras a, b, c, y d. Hay \(4!\) maneras de hacer esto:
        \[
            4! = 24
        \]

        Entre estas letras, hay 5 huecos donde se pueden colocar las 'e's (antes de la primera letra, entre las letras, y después de la última letra):
        \[
            \_ a \_ b \_ c \_ d \_
        \]

        Debemos colocar las 5 'e's en estos 5 huecos de modo que ninguna 'e' quede junto a otra. La única manera de hacer esto es elegir 5 de los 5 huecos disponibles, lo cual solo tiene 1 forma:
        \[
            \binom{5}{5} = 1
        \]

        Entonces, es simplemente el número de formas de ordenar a, b, c, y d:
        \[
            4! = 24
        \]
    \end{solution}

    \vspace{0.5cm}


    % Question 5
    \question \large\textbf{Un estudiante que realiza un examen debe responder 7 de las 10 preguntas. El orden no importa. ¿De cuántas formas puede responder el examen?}
    \begin{solution}
        El número de combinaciones de \(n\) elementos tomados de \(k\) en \(k\) está dado por el coeficiente binomial:
        \[
            \binom{n}{k} = \frac{n!}{k!(n-k)!}
        \]
        donde \(n!\) (n factorial) es el producto de todos los enteros positivos hasta \(n\), es decir, \(n! = n \times (n-1) \times (n-2) \times \ldots \times 1\).

        En este caso, queremos elegir 7 preguntas de un total de 10, por lo que \(n = 10\) y \(k = 7\):
        \[
            \binom{10}{7} = \frac{10!}{7!(10-7)!} = \frac{10!}{7!3!}
        \]

        Ahora calculamos \(10!\), \(7!\), y \(3!\):
        \[
            10! = 10 \times 9 \times 8 \times 7 \times 6 \times 5 \times 4 \times 3 \times 2 \times 1 = 3628800
        \]
        \[
            7! = 7 \times 6 \times 5 \times 4 \times 3 \times 2 \times 1 = 5040
        \]
        \[
            3! = 3 \times 2 \times 1 = 6
        \]

        Sustituyendo estos valores en la fórmula del coeficiente binomial:
        \[
            \binom{10}{7} = \frac{3628800}{5040 \times 6} = \frac{3628800}{30240} = 120
        \]

    \end{solution}

    \vspace{0.5cm}

    % Question 6
    \question \large\textbf{Juan quiere dar una fiesta para algunos de sus amigos. Debido al tamaño de su casa, sólo puede invitar a 11 de sus 20 amigos. ¿De cuántas formas puede seleccionar a los invitados?}
    \begin{solution}
        En este caso, queremos elegir 11 amigos de un total de 20, por lo que \(n = 20\) y \(k = 11\):
        \[
            \binom{20}{11} = \frac{20!}{11!(20-11)!} = \frac{20!}{11!9!}
        \]

        \[
            20! = 20 \times 19 \times 18 \times 17 \times 16 \times 15 \times 14 \times 13 \times 12 \times 11 \times 10 \times 9 \times 8 \times 7 \times 6 \times 5 \times 4 \times 3 \times 2 \times 1
        \]
        \[
            11! = 11 \times 10 \times 9 \times 8 \times 7 \times 6 \times 5 \times 4 \times 3 \times 2 \times 1 = 39916800
        \]
        \[
            9! = 9 \times 8 \times 7 \times 6 \times 5 \times 4 \times 3 \times 2 \times 1 = 362880
        \]

        Para evitar el cálculo directo de \(20!\), podemos simplificar usando la relación de factoriales:
        \[
            \binom{20}{11} = \frac{20 \times 19 \times 18 \times 17 \times 16 \times 15 \times 14 \times 13 \times 12 \times 11 \times 10}{11!}
        \]


        \[
            \binom{20}{11} = 167960
        \]
    \end{solution}

    \vspace{0.5cm}


    % Question 7
    \question \large\textbf{Un frasco contiene varias bolas rojas, azules y blancas. Suponiendo que las bolas del mismo color son indistinguibles y el orden importa, ¿cuántas secuencias diferentes pueden ocurrir si alguien dibuja...}
    \begin{enumerate}[label=\alph*.]
        \item 3 bolas y sale una de cada color?
        \item 5 bolas y obtiene 3 rojas y 2 azules?
        \item 5 bolas y obtiene exactamente 3 bolas rojas?
        \item 2 bolas, ambas del mismo color?
    \end{enumerate}
    \begin{solution}
        \begin{enumerate}[label=\alph*.]
            \item Para 3 bolas y sale una de cada color, hay \(3!\) permutaciones posibles:
                  \[
                      3! = 6
                  \]

            \item Para 5 bolas con 3 rojas y 2 azules, el número de secuencias diferentes es:
                  \[
                      \frac{5!}{3!2!} = 10
                  \]

            \item Para 5 bolas con exactamente 3 bolas rojas y las otras 2 de cualquier color, necesitamos considerar todas las combinaciones posibles. Si suponemos solo tres colores (rojo, azul, blanco), el conteo se hace más complicado y puede variar según la interpretación.

            \item Para 2 bolas del mismo color (suponiendo tres colores posibles), tenemos 3 combinaciones posibles:
                  \[
                      1 \text{ combinación por cada color} \times 3 \text{ colores} = 3
                  \]
        \end{enumerate}
    \end{solution}

    \vspace{0.5cm}

    % Question 8
    \question \large\textbf{Un club tiene 12 estudiantes de primer año y 8 de segundo año.}
    \begin{enumerate}[label=\alph*.]
        \item ¿De cuántas maneras podemos elegir un comité de 4 personas, 2 estudiantes de primer año y 2 estudiantes de segundo año?
        \item ¿De cuántas maneras podemos elegir un comité de 4 personas con al menos 2 estudiantes de primer año?
    \end{enumerate}
    \begin{solution}
        \begin{enumerate}[label=\alph*.]
            \item Número de maneras de elegir 2 estudiantes de primer año y 2 de segundo año:
                  \[
                      \binom{12}{2} \times \binom{8}{2} = 66 \times 28 = 1848
                  \]

            \item Para elegir un comité de 4 personas con al menos 2 estudiantes de primer año, consideramos los casos:
                  \begin{itemize}
                      \item 2 estudiantes de primer año y 2 de segundo año: \(66 \times 28 = 1848\)
                      \item 3 estudiantes de primer año y 1 de segundo año:
                            \[
                                \binom{12}{3} \times \binom{8}{1} = 220 \times 8 = 1760
                            \]
                      \item 4 estudiantes de primer año:
                            \[
                                \binom{12}{4} = 495
                            \]
                  \end{itemize}
                  Sumando todos los casos:
                  \[
                      1848 + 1760 + 495 = 4103
                  \]
        \end{enumerate}
    \end{solution}

    \vspace{0.5cm}

\end{questions}

\end{document}
