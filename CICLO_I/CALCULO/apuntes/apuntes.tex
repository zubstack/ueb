\documentclass[12pt,a4paper,fleqn]{article}
\usepackage[utf8]{inputenc}
\usepackage{amsmath}
\usepackage{amsfonts}
\usepackage{amssymb}
\usepackage{tcolorbox}
% \usepackage[showframe]{geometry}
\usepackage{geometry}
\usepackage{hyperref}
\usepackage{enumitem}
\usepackage{graphicx}
\usepackage{fancyhdr}

% \setlength{\mathindent}{0pt} % Ajusta la indentación a cero

% Márgenes
\geometry{top=2.5cm, bottom=2.5cm, left=2.5cm, right=2.5cm}

% Encabezado y pie de página
\pagestyle{fancy}
\fancyhead[L]{Apuntes de Derivadas}
\fancyhead[C]{}
\fancyhead[R]{\thepage}
\fancyfoot[L]{}
\fancyfoot[C]{}
\fancyfoot[R]{}

% Título
\title{Apuntes de Derivadas}
\author{Ariel Alejandro Calderón}
\date{\today}

\begin{document}

\maketitle

\newpage
\tableofcontents
\newpage

\section{Derivadas}
\subsection{Ejercicios}

\textbf{Hallar la derivada de $f(x)=x^3$}
\begin{equation*}
    f'(x) = \lim_{h \to 0} \dfrac{f(x+h) - f(x)}{h}
\end{equation*}
\begin{equation*}
    f'(x) = \lim_{h \to 0} \dfrac{(x+h)^3 - x^3}{h} = \dfrac{x^3 - x^3}{0} = \dfrac{0}{0}
\end{equation*}
\begin{equation*}
    f'(x) = \dfrac{x^3 + 3x^2h + 3xh^2 + h^3 - x^3}{h} = \dfrac{3x^2h + 3xh^2 + h^3}{h} = 3x^2 + 3xh + h^2
\end{equation*}
\begin{equation*}
    f'(x) = \lim_{h \to 0} (3x^2 + 3xh + h^2) = 3x^2
\end{equation*}\\[10pt]
% Ejercicio
\textbf{Hallar la derivada de $f(x)=\sqrt[3]{x^2}$}
\begin{equation*}
    f(x) = x^{\frac{2}{3}} \implies f'(x)= \frac{2}{3}x^{\frac{2}{3}-1}
\end{equation*}
\begin{equation*}
    f'(x)= \frac{2}{3}x^{-\frac{1}{3}}=\frac{2}{3}\cdot\frac{1}{\sqrt[3]{x}}=\frac{2}{3\sqrt[3]{x}}
\end{equation*}\\[10pt]
% Ejercicio
\textbf{Hallar la derivada de $f(x)= \sin x$}\\[10pt]
\boxed{\sin A - \sin B = 2 \sin \left(\frac{A-B}{2}\right)\cos \left(\frac{A+B}{2}\right)}

\begin{align*}
    f'(x) & =\lim_{h\to0}\dfrac{\sin (x+h)-\sin x}{h}                                \\
          & =\lim_{h\to0}\dfrac{2\sin (\frac{h}{2})\cos (2+\frac{h}{2})}{h}          \\
          & =\lim_{h\to0}\dfrac{\sin (\frac{h}{2})\cos (2+\frac{h}{2})}{\frac{h}{2}} \\
          & =\lim_{h\to0}1\cdot\cos(x+\frac{h}{2})                                   \\
          & =\cos x
\end{align*}\\[10pt]
% Ejercicio
\textbf{Hallar la derivada de $f(x)= e^x$}
\begin{align*}
    f'(x) & = \lim_{h \to 0} \dfrac{e^{x+h} - e^x}{h}       \\
          & = \lim_{h \to 0} \dfrac{e^x \cdot e^h - e^x}{h} \\
          & = \lim_{h \to 0} e^x \cdot \dfrac{e^h - 1}{h}   \\
          & = e^x \cdot \lim_{h \to 0} \dfrac{e^h - 1}{h}   \\
          & = e^x \cdot 1 \text{ *}                         \\
          & = e^x
\end{align*}
% Ejercicio
\textbf{Hallar la derivada de $f(x)= \ln x$}
\begin{align*}
    f'(x) & = \lim_{h \to 0} \dfrac{\ln{x+h} - \ln x}{h}      \\
          & = \lim_{h \to 0} \dfrac{\ln{\frac{x+h}{x}}}{h}    \\
          & \dfrac{h}{x}=u \implies h=ux                      \\
          & = \lim_{u \to 0} \dfrac{\ln (1+u)}{ux}            \\
          & = \dfrac{1}{x}\lim_{u \to 0} \dfrac{\ln (1+u)}{u} \\
          & = \dfrac{1}{x}\cdot 1 = \dfrac{1}{x}              \\
\end{align*}

\begin{tcolorbox}[colback=white!95!blue, colframe=blue!40!black, title=Teoremas de Álgebra de Derivadas]
    \[  
        (f \pm g)'(x) = f'(x) \pm g'(x) 
    \]\
    \[
        (f \cdot g)'(x) = f'(x) g(x) + f(x) g'(x) 
    \]\
    \[
        \left( \frac{f}{g} \right)'(x) = \frac{f'(x) g(x) - f(x) g'(x)}{g(x)^2}, \quad g(x) \neq 0
    \]
\end{tcolorbox}


\end{document}
