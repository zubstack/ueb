\documentclass[answers]{exam} % Clase para exámenes con respuestas
\usepackage[english,spanish]{babel} % Soporte para inglés y español
\usepackage[autostyle]{csquotes} % Manejo de citas
\usepackage{amsmath, amssymb} % Paquetes para matemáticas avanzadas
\usepackage{graphicx} % Inclusión de gráficos
\usepackage{enumitem} % Personalización de listas enumeradas
\usepackage[letterpaper,top=2cm,bottom=2cm,left=3cm,right=3cm,marginparwidth=1.75cm]{geometry} % Configuración de márgenes
\usepackage[colorlinks=true, allcolors=blue]{hyperref} % Enlaces con color

\renewcommand{\solutiontitle}{\noindent\textbf{Respuesta:}\par\noindent} % Personalización del título de respuestas
\renewcommand{\familydefault}{\sfdefault}

% Configuración de encabezado y pie de página
\pagestyle{headandfoot}
\firstpageheader{Universidad de Bolívar}{}{\selectlanguage{spanish}\today} 
\runningheader{Universidad de Bolívar}{}{Cálculo I}
\firstpagefooter{}{\thepage}{}
\runningfooter{}{\thepage}{}

\begin{document}

\begin{center}
	\large\textbf{Trabajo Autónomo 2.10 - Cálculo I}\\[1em]
	\large Primer Ciclo \enquote*{A} - Ingeniería de Software\\[1em]
\end{center}

\vspace{0.5cm}
\large\textbf{Estudiante:} Ariel Alejandro Calderón
\vspace{0.5cm}

\begin{questions}

	% Pregunta 1
	\question \large\textbf{Calcular los siguientes límites:}
	\begin{enumerate}[label=\alph*.]
		\item $\displaystyle \lim_{x\to{0}} \frac{4x-3x^2+8x^3}{2x-5x^2}$
		\item $\displaystyle \lim_{x\to{5}} \frac{3-\sqrt{4+x}}{x-5}$
		\item $\displaystyle \lim_{x\to{1}} \frac{x^3+x^2-5x+3}{x^4-x^3-x^2+x}$
		\item $\displaystyle \lim_{x\to{2}} \frac{x-3}{x+4}$
	\end{enumerate}
	\begin{solution}

	\end{solution}

	\vspace{0.5cm}

	% Pregunta 2
	\question \large\textbf{Calcular $\displaystyle \lim_{x\to{1}} |x-1|$}
	\begin{solution}
	\end{solution}

	\newpage

	% Pregunta 3
	\question \large\textbf{Calcular $\displaystyle \lim_{x\to{0}} \frac{x}{|x|}$}
	\begin{solution}
	\end{solution}

	\vspace{0.5cm}

	% Pregunta 4
	\question \large\textbf{Si $ f(x) =
			\begin{cases}
				x^2+2, & \text{si } x \geq 1 \\
				x+b,   & \text{si } x < 1
			\end{cases}
		$ donde a y b son constantes. \\[1em] Para que exista $\displaystyle \lim_{x\to{1}} f(x)$, ¿qué relación debe haber entre a y b?}
	\begin{solution}
	\end{solution}

	\vspace{0.5cm}

	% Pregunta 5
	\question \large\textbf{Calcular $\displaystyle \lim_{x\to{1}} \frac{|x^2-1|}{x-1}$}
	\begin{solution}
	\end{solution}

	\vspace{0.5cm}

	% Pregunta 6
	\question \large\textbf{Calcular $\displaystyle \lim_{x\to{0}} f(x) \text{ si } f(x) =
			\begin{cases}
				x^3, &  \text{si } x \leq 0 \\
				x^2, &  \text{si } x > 0
			\end{cases}$}
	\begin{solution}
	\end{solution}

	\vspace{0.5cm}

	% Pregunta 7
	\question \large\textbf{Calcular los siguientes límites:}
	\begin{enumerate}[label=\alph*.]
		\item $\displaystyle \lim_{x\to{-\infty}} \frac{x^2-5x+3}{-3x^2+x-2}$
		\item $\displaystyle \lim_{x\to{-\infty}} (\sqrt{x^2+5}-\sqrt{x}+2)$
		\item $\displaystyle \lim_{x\to{\infty}} (\sqrt{x^2+3x}-\sqrt{x^2+x})$
		\item $\displaystyle \lim_{x\to{\infty}} (\sqrt{x^2+1}-\sqrt{x^2-4x})$
	\end{enumerate}
	\begin{solution}
	\end{solution}

	\vspace{0.5cm}

	% Pregunta 8
	\question \large\textbf{Calcular los siguientes límites trigonométricos:}
	\begin{enumerate}[label=\alph*.]
		\item $\displaystyle \lim_{x\to{0}} \frac{1- cos^2 x}{x}$
		\item $\displaystyle \lim_{x\to{0}} \frac{cos^2 x}{x}$
		\item $\displaystyle \lim_{x\to{-1}} \frac{sen(x+1)}{x+1}$
		\item $\displaystyle \lim_{x\to{0}} \frac{sen(3x)}{x}$
		\item $\displaystyle \lim_{x\to{0}} \frac{sen(3x)}{sen(2x)}$
		\item $\displaystyle \lim_{x\to{\pi}} x * secx$
	\end{enumerate}
	\begin{solution}
	\end{solution}

	\vspace{0.5cm}

\end{questions}

\end{document}
