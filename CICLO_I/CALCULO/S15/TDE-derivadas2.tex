\documentclass[answers]{exam} % Clase para exámenes con respuestas
\usepackage[english,spanish]{babel} % Soporte para inglés y español
\usepackage[autostyle]{csquotes} % Manejo de citas
\usepackage{amsmath, amssymb} % Paquetes para matemáticas avanzadas
\usepackage{graphicx} % Inclusión de gráficos
\usepackage{enumitem} % Personalización de listas enumeradas
\usepackage[letterpaper,top=2cm,bottom=2cm,left=3cm,right=3cm,marginparwidth=1.75cm]{geometry} % Configuración de márgenes
\usepackage[colorlinks=true, allcolors=blue]{hyperref} % Enlaces con color

\usepackage{array}   % for adjusting row height
\renewcommand{\arraystretch}{1.5} % adjust the vertical spacing between rows

\renewcommand{\solutiontitle}{\noindent\textbf{Respuesta:}\par\noindent} % Personalización del título de respuestas
\renewcommand{\familydefault}{\sfdefault}

% Configuración de encabezado y pie de página
\pagestyle{headandfoot}
\firstpageheader{Universidad de Bolívar}{}{24 de julio del 2024} 
\runningheader{Universidad de Bolívar}{}{Calculo I}
\firstpagefooter{}{\thepage}{}
\runningfooter{}{\thepage}{}

\begin{document}

\begin{center}
    \large\textbf{Trabajo Autónomo 2.14 - Cálculo I}\\[1em]
    \large Primer Ciclo \enquote*{A} - Ingeniería de Software\\[1em]
\end{center}

\vspace{0.5cm}
\large\textbf{Estudiante:} Ariel Alejandro Calderón
\vspace{0.5cm}

\begin{questions}

    % Question 1
    \question \large\textbf{Presentar una tabla con las fórmulas de las derivadas de las funciones básicas:}

    \begin{center}


        \noindent
        \begin{minipage}{0.5\textwidth}
            \centering
            \begin{tabular}{|>{\centering\arraybackslash}m{3.5cm}|>{\centering\arraybackslash}m{3cm}|}
                \hline
                Función           & Derivada                       \\
                \hline
                $(k)$ (constante) & $0$                            \\
                $x^n$             & $nx^{n-1}$                     \\
                $x^\frac{m}{n}$   & $\frac{m}{n}x^{\frac{m}{n}-1}$ \\
                $\sin(x)$         & $\cos(x)$                      \\
                $\cos(x)$         & $-\sin(x)$                     \\
                $\tan(x)$         & $\sec^2(x)$                    \\
                $\arcsin(x)$      & $\dfrac{1}{\sqrt{1-x^2}}$      \\
                $\arccos(x)$      & $-\dfrac{1}{\sqrt{1-x^2}}$     \\
                $\arctan(x)$      & $\dfrac{1}{1+x^2}$             \\
                $e^x$             & $e^x$                          \\
                $\ln(x)$          & $\dfrac{1}{x}$                 \\[10px]
                \hline
            \end{tabular}
        \end{minipage}%
        \begin{minipage}{0.5\textwidth}
            \centering
            \begin{tabular}{|>{\centering\arraybackslash}m{3.5cm}|>{\centering\arraybackslash}m{3cm}|}
                \hline
                $a^x$              & $a^x \ln(a)$                    \\
                $\log_b{x}$        & $\dfrac{1}{x \ln(b)}$           \\
                $\sin^{-1}(x)$     & $\dfrac{1}{\sqrt{1-x^2}}$       \\
                $\cos^{-1}(x)$     & $-\dfrac{1}{\sqrt{1-x^2}}$      \\
                $\tan^{-1}(x)$     & $\dfrac{1}{1+x^2}$              \\
                $\cot^{-1}(x)$     & $-\dfrac{1}{1+x^2}$             \\
                $\sec^{-1}(x)$     & $\dfrac{1}{|x|\sqrt{x^2-1}}$    \\
                $\csc^{-1}(x)$     & $-\dfrac{1}{|x|\sqrt{x^2-1}}$   \\
                $ax^n$             & $anx^{n-1}$                     \\
                $ax^{\frac{m}{n}}$ & $\frac{m}{n}ax^{\frac{m}{n}-1}$ \\
                $ax^{bx}$          & $abx^{bx} \ln(a)$               \\[9px]
                \hline
            \end{tabular}
        \end{minipage}


    \end{center}

    \vspace{0.5cm}

    % Question 2
    \question \large\textbf{Resolver los siguientes ejercicios de derivadas: }

    \begin{enumerate}[label=\alph*.]
        \item $\displaystyle f(x)=(\sin(x) + x)^2$
              \[
                  f'(x) = 2(\sin(x) + x)(\cos(x) + 1)
              \]

        \item $\displaystyle f(x)=\log(x^2 + 2x^4)$
              \[
                  f'(x) = \frac{1}{x^2 + 2x^4} \cdot (2x + 8x^3) = \frac{2x(1 + 4x^2)}{x^2 + 2x^4}
              \]

        \item $\displaystyle f(x)=a^{x^2}, a \text{ constante}$
              \[
                  f'(x) = a^{x^2} \cdot \ln(a) \cdot 2x
              \]

        \item $\displaystyle f(x)=a^{\tan(nx)}, a \text{ constante}$
              \[
                  f'(x) = a^{\tan(nx)} \cdot \ln(a) \cdot \sec^2(nx) \cdot n
              \]

        \item $\displaystyle f(x)=\arcsin\left(\dfrac{x}{\sqrt{1+x^2}}\right)$
              \[
                  f'(x) = \frac{1}{\sqrt{1-\left(\frac{x}{\sqrt{1+x^2}}\right)^2}} \cdot \left( \frac{\sqrt{1+x^2} - \frac{x \cdot x}{\sqrt{1+x^2}}}{(1+x^2)} \right) = \frac{1}{\sqrt{1 - \frac{x^2}{1+x^2}}} \cdot \frac{1}{\sqrt{1+x^2}} \\
              \]
              \[
                  = \frac{1+x^2}{\sqrt{1+x^2-x^2}} \cdot \frac{1}{\sqrt{1+x^2}} = \frac{1}{1+x^2}
              \]


        \item $\displaystyle f(x)=x^x$
              \[
                  f'(x) = x^x (\ln(x) + 1)
              \]

        \item $\displaystyle f(x)=\tan^{-1}(x^2)$
              \[
                  f'(x) = \frac{2x}{1+(x^2)^2} = \frac{2x}{1+x^4}
              \]
    \end{enumerate}


    \vspace{0.5cm}

    % Question 3
    \question \large\textbf{Resolver los siguientes ejercicios de derivadas: }


    \begin{enumerate}[label=\alph*.]
        \item $\displaystyle 2x^3-4x^2y+y^2=0$
              \[
                  \frac{d}{dx}(2x^3 - 4x^2y + y^2) = 0 \implies 6x^2 - 4x^2 \frac{dy}{dx} - 8xy + 2y \frac{dy}{dx} = 0
              \]
              \[
                  \implies 6x^2 - 8xy = (4x^2 - 2y) \frac{dy}{dx}
              \]
              \[
                  \implies \frac{dy}{dx} = \frac{6x^2 - 8xy}{4x^2 - 2y}
              \]
        \item $\displaystyle e^{x}\sin(y)+e^{y}\cos(x)=1$
              \[
                  \frac{d}{dx}(e^x \sin(y) + e^y \cos(x)) = 0 \implies e^x \sin(y) + e^y (-\sin(x)) \frac{dy}{dx} + e^y \cos(x) + e^x \cos(x) \frac{dy}{dx} = 0
              \]
              \[
                  \implies e^x \sin(y) + e^y \cos(x) = -e^y \sin(x) \frac{dy}{dx} - e^x \cos(x) \frac{dy}{dx} \\[10pt]
              \]
              \[
                  \implies \frac{dy}{dx} = \frac{e^x \sin(y) + e^y \cos(x)}{-e^y \sin(x) - e^x \cos(x)}
              \]
    \end{enumerate}
    \vspace{0.5cm}

    % Question 4
    \question \large\textbf{Encontrar las derivadas de orden superior que se indican: }

    \begin{enumerate}[label=\alph*.]
        \item $\displaystyle y=10x^2-3x+1,\text{ } y''$
              \[
                  y' = 20x - 3
              \]
              \[
                  y'' = 20
              \]

        \item $\displaystyle y=\sin(7x),\text{ } y''$
              \[
                  y' = 7\cos(7x)
              \]
              \[
                  y'' = -49\sin(7x)
              \]

        \item $\displaystyle Ax^2+Bxy+Cy^2+Dx+Ey+F=1, A,B,C,D,E,F \text{ son constantes. } y''$
              \[
                  \frac{d}{d x}(2Ax + By + D) + \frac{d}{d y}(2Cy + Bx + E) = 0
              \]
              \[
                  y'' = -\frac{2A + By'}{B + 2C y'}
              \]
    \end{enumerate}


    \vspace{0.5cm}

    % Question 5
    \question \large\textbf{Encontrar las derivadas parciales que se indican: }


    \begin{enumerate}[label=\alph*.]
        \item $\displaystyle f(x,y)=-x^2+2xy-y$ ,encontrar $f'_x$ y $f'_y$
              \[
                  f'_x = \frac{d}{d x} (-x^2 + 2xy - y) = -2x + 2y
              \]
              \[
                  f'_y = \frac{d}{d y} (-x^2 + 2xy - y) = 2x - 1
              \]

        \item $\displaystyle f(x,y)=\sqrt{x^3+y^2}$, calcular $f'_x(1,1)$
              \[
                  f'_x = \frac{d}{d x} (\sqrt{x^3 + y^2}) = \frac{1}{2\sqrt{x^3 + y^2}} \cdot 3x^2 = \frac{3x^2}{2\sqrt{x^3 + y^2}}
              \]
              \[
                  f'_x(1,1) = \frac{3 \cdot 1^2}{2 \sqrt{1^3 + 1^2}} = \frac{3}{2\sqrt{2}}
              \]

        \item $\displaystyle f(x,y)=\dfrac{2xy-y}{x^2+y}$, encontrar $f'_x$ y $f'_y$
              \begin{align*}
                  f'_x & = \frac{d}{d x} \left( \frac{2xy - y}{x^2 + y} \right) = \frac{(2y)(x^2 + y) - (2xy - y)(2x)}{(x^2 + y)^2} \\
                       & = \frac{2yx^2 + 2y^2 - 4x^2y + 2xy}{(x^2 + y)^2}                                                           \\
                       & = \frac{-2yx^2 + 2y^2 + 2xy}{(x^2 + y)^2}
              \end{align*}
              \begin{align*}
                  f'_y & = \frac{d}{d y} \left( \frac{2xy - y}{x^2 + y} \right) = \frac{(2x - 1)(x^2 + y) - (2xy - y)}{(x^2 + y)^2} \\
                       & = \frac{2x x^2 + 2xy - x^2 - y - 2xy + y}{(x^2 + y)^2}                                                     \\
                       & = \frac{2xx^2 - x^2}{(x^2 + y)^2}                                                                          \\
                       & = \frac{x(2x - 1)}{(x^2 + y)^2}
              \end{align*}
    \end{enumerate}

    \vspace{0.5cm}
    \newpage
    % Question 6
    \question \large\textbf{Encontrar lo que se indica: }

    \begin{enumerate}[label=\alph*.]
        \item $\displaystyle f'\left(\dfrac{\pi}{4}\right)$ si $f(x)=\sin(x)+x$
              \[
                  f'(x) = \cos(x) + 1
              \]
              \[
                  f'\left(\dfrac{\pi}{4}\right) = \cos\left(\dfrac{\pi}{4}\right) + 1 = \dfrac{\sqrt{2}}{2} + 1
              \]

        \item $\displaystyle f'(2)$ si $f(x)=x^x$
              \[
                  f(x) = e^{x \ln(x)}
              \]
              \[
                  f'(x) = e^{x \ln(x)} \left( \ln(x) + 1 \right) = x^x \left( \ln(x) + 1 \right)
              \]
              \[
                  f'(2) = 2^2 \left( \ln(2) + 1 \right) = 4 \left( \ln(2) + 1 \right)
              \]

        \item Encontrar la ecuación de la recta tangente a $f(x)=x^2-2x+2$ en $x=2$
              \[
                  f'(x) = 2x - 2
              \]
              \[
                  f'(2) = 2(2) - 2 = 2
              \]
              La pendiente de la recta tangente es $2$ y el punto tangente es:
              \[
                  (2, f(2)) = (2, 2^2 - 2(2) + 2) = (2, 2)
              \]

              La ecuación de la recta tangente es:
              \[
                  y - 2 = 2(x - 2) \implies y = 2x - 2
              \]
    \end{enumerate}
    \vspace{0.5cm}
\end{questions}

\end{document}