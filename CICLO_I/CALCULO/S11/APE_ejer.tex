\documentclass[answers]{exam} % Clase para exámenes con respuestas
\usepackage[english,spanish]{babel} % Soporte para inglés y español
\usepackage[autostyle]{csquotes} % Manejo de citas
\usepackage{amsmath, amssymb} % Paquetes para matemáticas avanzadas
\usepackage{graphicx} % Inclusión de gráficos
\usepackage{enumitem} % Personalización de listas enumeradas
\usepackage[letterpaper,top=2cm,bottom=2cm,left=3cm,right=3cm,marginparwidth=1.75cm]{geometry} % Configuración de márgenes
\usepackage[colorlinks=true, allcolors=blue]{hyperref} % Enlaces con color

\renewcommand{\solutiontitle}{\noindent\textbf{Respuesta:}\par\noindent} % Personalización del título de respuestas
\renewcommand{\familydefault}{\sfdefault}

% Configuración de encabezado y pie de página
\pagestyle{headandfoot}
\firstpageheader{Universidad de Bolívar}{}{\selectlanguage{spanish}\today} 
\runningheader{Universidad de Bolívar}{}{Cálculo I}
\firstpagefooter{}{\thepage}{}
\runningfooter{}{\thepage}{}

\begin{document}

% \begin{center}
%     \large\textbf{<<TEMA>> - <<ASIGNAUTURA>>}\\[1em]
%     \large Primer Ciclo \enquote*{A} - Ingeniería de Software\\[1em]
% \end{center}

% \vspace{0.5cm}
% \large\textbf{Estudiante:} Ariel Alejandro Calderón
% \vspace{0.5cm}
\large\textbf{Funciones:}
\[
    f(x)=\frac{3-\sqrt{x^2-7}}{x-4} \text{ si } x \rightarrow 4
\]


\begin{questions}

    % Question 1
    \question \large\textbf{Para la primera función f(x) encontrar matemáticamente, el límite indicado}
    \begin{solution}
        A lot!
    \end{solution}

    \vspace{0.5cm}


\end{questions}

\end{document}
