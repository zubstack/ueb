\documentclass[answers]{exam} % Clase para exámenes con respuestas
\usepackage[spanish,es-tabla]{babel} % Soporte para español y tablas
\usepackage[autostyle]{csquotes} % Manejo de citas
\usepackage{amsmath, amssymb} % Paquetes para matemáticas avanzadas
\usepackage{graphicx} % Inclusión de gráficos
\usepackage{enumitem} % Personalización de listas enumeradas
\usepackage[letterpaper,top=2cm,bottom=2cm,left=3cm,right=3cm,marginparwidth=1.75cm]{geometry} % Configuración de márgenes
\usepackage[colorlinks=true, allcolors=blue]{hyperref} % Enlaces con color

\renewcommand{\solutiontitle}{\noindent\textbf{Respuesta:}\par\noindent} % Personalización del título de respuestas
\renewcommand{\familydefault}{\sfdefault}
% Ajustando el \arraystretch para aumentar la altura de las filas
\renewcommand{\arraystretch}{1.5}

% Configuración de encabezado y pie de página
\pagestyle{headandfoot}
\firstpageheader{Universidad de Bolívar}{}{\today} 
\runningheader{Universidad de Bolívar}{}{Álgebra Lineal}
\firstpagefooter{}{\thepage}{}
\runningfooter{}{\thepage}{}

\begin{document}

\begin{center}
	\large\textbf{Trabajo autónomo - Álgebra Lineal}\\[1em]
	\large Primer Ciclo \enquote*{A} - Ingeniería de Software\\[1em]
\end{center}

\vspace{0.5cm}
\large\textbf{Estudiante:} Ariel Alejandro Calderón
\vspace{0.5cm}

\begin{questions}

	% Pregunta 1
	\question \large\textbf{Escribe la matriz aumentada de este sistema de ecuaciones lineales:
		\[
			\left\{
			\begin{array}{l}
				3x_1 + 4x_2 = 10 \\
				x_1 + 5x_3 = 15  \\
				-x_2 + x_3 = 20
			\end{array}
			\right.
		\]}
	\begin{solution}
		\[
			\left[
				\begin{array}{ccc|c}
					3 & 4  & 0 & 10 \\
					1 & 0  & 5 & 15 \\
					0 & -1 & 1 & 20
				\end{array}
				\right]
		\]
	\end{solution}

	\vspace{0.5cm}

	% Pregunta 2
	\question \large\textbf{Escribe el sistema de ecuaciones lineales representado por la matriz aumentada:
		\[
			\left[
				\begin{array}{ccc|c}
					{1}  & {0} & {-1} & 1  \\
					{-1} & {1} & {0}  & 3  \\
					{0}  & {2} & {1}  & -5 \\
				\end{array}
				\right]\]
	}
	\begin{solution}
		\[
			\left\{
			\begin{array}{l}
				x_1 - x_3 = 1  \\
				-x_1 + x_2 = 3 \\
				2x_2 + x_3 = -5
			\end{array}
			\right.
		\]
	\end{solution}

	\vspace{0.5cm}

	% Pregunta 3
	\question \large\textbf{Si una matriz está en forma reducida, indícalo. Si no lo está, explica por qué e indica la(s) operación(es) necesaria(s) para transformar la matriz a forma reducida:}

	\begin{enumerate}[label=\alph*.]
		\item $\displaystyle
			      \left[
				      \begin{array}{ccc|c}
					      {1} & {0} & {0} & -2 \\
					      {0} & {1} & {0} & 0  \\
					      {0} & {0} & {1} & 1  \\
				      \end{array}
				      \right]
		      $
		      \vspace{0.5cm}
		      \textbf{Respuesta:} La matriz está en forma reducida.
		\item $\displaystyle
			      \left[
				      \begin{array}{ccc|c}
					      {1} & {0} & {-1} & 3 \\
					      {0} & {2} & {1}  & 1 \\
					      {0} & {0} & {0}  & 0 \\
				      \end{array}
				      \right]
		      $
		      \vspace{0.5cm}
		      \textbf{Respuesta:} No está en forma reducida. Realizar las siguientes operaciones:
		      \begin{itemize}
			      \item Fila 1: Sumar Fila 2 a Fila 1
			      \item Fila 2: Dividir Fila 2 por 2
		      \end{itemize}
	\end{enumerate}

	\vspace{0.5cm}
	% Pregunta 4
	\question \large\textbf{Escribe el sistema lineal correspondiente a cada matriz aumentada reducida y escribe la solución del sistema:}

	\begin{enumerate}[label=\alph*.]
		\item $\displaystyle
			      \left[
				      \begin{array}{cccc|c}
					      {1} & {0} & {0} & {0} & -2 \\
					      {0} & {1} & {0} & {0} & 0  \\
					      {0} & {0} & {1} & {0} & 1  \\
					      {0} & {0} & {0} & {1} & 3  \\
				      \end{array}
				      \right]
		      $
		      \vspace{0.5cm}
		      \begin{solution}
			      \[
				      \left\{
				      \begin{array}{l}
					      x_1 = -2 \\
					      x_2 = 0  \\
					      x_3 = 1  \\
					      x_4 = 3
				      \end{array}
				      \right.
			      \]
		      \end{solution}


		\item $\displaystyle
			      \left[
				      \begin{array}{cccc|c}
					      {1} & {0} & {-2} & {3} & 4 \\
					      {0} & {1} & {-1} & {2} & 1 \\
				      \end{array}
				      \right]
		      $
		      \vspace{0.5cm}
		      \begin{solution}
			      \[
				      \left\{
				      \begin{array}{l}
					      x_1 = 4 + 2s - 3t \\
					      x_2 = 1 + s - 2t  \\
					      x_3 = s           \\
					      x_4 = t
				      \end{array}
				      \right.
			      \]
			      donde \( s \) y \( t \) son parámetros arbitrarios.
		      \end{solution}

	\end{enumerate}

	\vspace{0.5cm}
	% Pregunta 5
	\question \large\textbf{¿En cuál de los Problemas 20, 22, 24, 26 y 28 el número de unos más a la izquierda es menor que el número de variables?:}


	\begin{solution}
		22, 26, 28
	\end{solution}

	\vspace{0.5cm}
	% Pregunta 6
	\question \large\textbf{Si el número de unos más a la izquierda es menor que el número de variables y el sistema es consistente, entonces el sistema tiene infinitas soluciones:}


	\begin{solution}
		Cierto
	\end{solution}

	\vspace{0.5cm}
	% Pregunta 7
	\question \large\textbf{Usa operaciones de fila para cambiar cada matriz a forma reducida:
		\[
			\left[
				\begin{array}{ccc|c}
					{1} & {1} & {1} & 8  \\
					{3} & {5} & {7} & 30 \\
				\end{array}
				\right]
		\]
	}

	\begin{solution}
		\[
			R_2 - 3R_1 \rightarrow R_2
			\left[
				\begin{array}{ccc|c}
					{1} & {1} & {1} & 8 \\
					{0} & {2} & {4} & 6 \\
				\end{array}
				\right]
		\]
		\[
			\frac{1}{2}R_2 \rightarrow R_2
			\left[
				\begin{array}{ccc|c}
					{1} & {1} & {1} & 8 \\
					{0} & {1} & {2} & 3 \\
				\end{array}
				\right]
		\]
		\[
			R_1 - R_2 \rightarrow R_1
			\left[
				\begin{array}{ccc|c}
					{1} & {0} & {-1} & 5 \\
					{0} & {1} & {2}  & 3 \\
				\end{array}
				\right]
		\]
	\end{solution}

	\vspace{0.5cm}

	% Pregunta 8
	\question \large\textbf{Resuelve usando eliminación de Gauss–Jordan:}

	\begin{enumerate}[label=\alph*.]
		\item $\displaystyle
			      \left\{
			      \begin{array}{l}
				      3x_1 + 5x_2 - x_3 = -7 \\
				      x_1 + x_2 + x_3 = -1   \\
				      2x_1 + 11x_3 = 7
			      \end{array}
			      \right.
		      $
		      \vspace{0.5cm}

		      \begin{solution}
			      \[
				      \left[
					      \begin{array}{ccc|c}
						      3 & 5 & -1 & -7 \\
						      1 & 1 & 1  & -1 \\
						      2 & 0 & 11 & 7
					      \end{array}
					      \right]
			      \]

			      \[
				      R_2 - \frac{1}{3} R_1 \rightarrow R_2
				      \left[
					      \begin{array}{ccc|c}
						      3 & 5            & -1          & -7          \\
						      0 & -\frac{2}{3} & \frac{4}{3} & \frac{4}{3} \\
						      2 & 0            & 11          & 7
					      \end{array}
					      \right]
			      \]

			      \[
				      R_3 - \frac{2}{3} R_1 \rightarrow R_3
				      \left[
					      \begin{array}{ccc|c}
						      3 & 5             & -1           & -7           \\
						      0 & -\frac{2}{3}  & \frac{4}{3}  & \frac{4}{3}  \\
						      0 & -\frac{10}{3} & \frac{35}{3} & \frac{35}{3}
					      \end{array}
					      \right]
			      \]

			      \[
				      R_3 - 5 R_2 \rightarrow R_3
				      \left[
					      \begin{array}{ccc|c}
						      3 & 5            & -1          & -7          \\
						      0 & -\frac{2}{3} & \frac{4}{3} & \frac{4}{3} \\
						      0 & 0            & 5           & 5
					      \end{array}
					      \right]
			      \]

			      \[
				      \begin{cases}
					      3x_1 + 5x_2 - x_3 = -7 \\
					      x_1 + x_2 + x_3 = -1   \\
					      2x_1 + 11x_3 = 7
				      \end{cases}
			      \]

			      \[
				      \begin{aligned}
					      5x_3             & = 5 \rightarrow x_3 = 1            \\
					      -\frac{2}{3} x_2 & = \frac{4}{3} - \frac{4}{3}x_3 = 0 \\
					      x_1              & = -7 - 5x_2 + x_3 = -6
				      \end{aligned}
			      \]

			      \[
				      \begin{aligned}
					      x_1 & = -6 \\
					      x_2 & = 0  \\
					      x_3 & = 1
				      \end{aligned}
			      \]
		      \end{solution}
			  \newpage

		\item $\displaystyle
			      \left\{
			      \begin{array}{l}
				      2x_1 - x_2 = 0  \\
				      3x_1 + 2x_2 = 7 \\
				      x_1 - x_2 = -2
			      \end{array}
			      \right.
		      $
		      \vspace{0.5cm}

		      \begin{solution}
			      \[
				      \left[
					      \begin{array}{cc|c}
						      2 & -1 & 0  \\
						      3 & 2  & 7  \\
						      1 & -1 & -2
					      \end{array}
					      \right]
			      \]
			      \[
				      R_2 - \frac{3}{2}R_1 \rightarrow R_2
				      \left[
					      \begin{array}{cc|c}
						      2 & -1          & 0  \\
						      0 & \frac{7}{2} & 7  \\
						      0 & -1          & -2
					      \end{array}
					      \right]
			      \]
			      \[
				      R_3 - \frac{1}{2}R_1 \rightarrow R_3
				      \left[
					      \begin{array}{cc|c}
						      2 & -1           & 0  \\
						      0 & \frac{7}{2}  & 7  \\
						      0 & \frac{-1}{2} & -2
					      \end{array}
					      \right]
			      \]
			      \[
				      R_3 - \frac{-1}{7}R_2 \rightarrow R_3
				      \left[
					      \begin{array}{cc|c}
						      2 & -1          & 0  \\
						      0 & \frac{7}{2} & 7  \\
						      0 & 0           & -1
					      \end{array}
					      \right]
			      \]
			      \[
				      \begin{cases}
					      2x_1 - x_2 = 0     \\
					      \frac{7}{2}x_2 = 7 \\
					      0 = -1
				      \end{cases}
			      \]

			      No hay soluciones.
		      \end{solution}
	\end{enumerate}

\end{questions}

\end{document}
