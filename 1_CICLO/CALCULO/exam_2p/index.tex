\documentclass[answers]{exam}
\usepackage[english]{babel}
\usepackage[letterpaper,top=2cm,bottom=2cm,left=3cm,right=3cm,marginparwidth=1.75cm]{geometry}
\usepackage{amsmath, amssymb}
\usepackage{graphicx}
\usepackage[letterpaper,top=2cm,bottom=2cm,left=3cm,right=3cm,marginparwidth=1.75cm]{geometry} 
\usepackage{array}  
\renewcommand{\arraystretch}{1.5} 
\renewcommand{\solutiontitle}{\noindent\textbf{Respuesta:}\par\noindent} %
\renewcommand{\familydefault}{\sfdefault}

% Header and footer
\pagestyle{headandfoot}
\firstpageheader{Universidad de Bolívar}{}{\today}
\runningheader{Universidad de Bolívar}{}{\thepage}
\firstpagefooter{}{}{}
\runningfooter{}{}{}

\begin{document}

\begin{center}
	\Large\textbf{Corrección del Examen II}\\[1em]
	\large\textbf{Cálculo I}\\[1em]
	\large Primer Ciclo "A" - Ingeniería de Software\\[1em]
	% \large \today
\end{center}


\begin{questions}

	\question \large\textbf{Determinar si la  función \(f(x) =
		\begin{cases}
			3x-5,                    & \text{si } x \le 1 \\[1em]
			\dfrac{x^2-1}{x^2-3x+2}, & \text{si } x > 1
		\end{cases}\) es continua en $x=1$}

	\vspace{0.5cm}

	Evaluamos las tres condiciones de continuidad:\\

	(I). \textbf{Existencia de \( f(1) \):  }
	Dado que \( x = 1 \) pertenece al dominio de la primera rama, evaluamos:
	\[
		f(1) = 3(1) - 5 = -2.
	\]

	(II). \textbf{Existencia del límite cuando \( x \to 1 \):  }
	Evaluamos el límite lateral izquierdo y derecho:\\

	- Límite cuando \( x \to 1^- \) (usando la primera rama):
	\[
		\lim_{x \to 1^-} f(x) = \lim_{x \to 1^-} (3x - 5) = 3(1) - 5 = -2.
	\]
	- Límite cuando \( x \to 1^+ \) (usando la segunda rama):
	\[
		\lim_{x \to 1^+} f(x) = \lim_{x \to 1^+} \dfrac{x^2 - 1}{x^2 - 3x + 2}.
	\]
	\[
		\dfrac{x^2 - 1}{x^2 - 3x + 2} =  \dfrac{(x - 1)(x + 1)}{(x - 1)(x - 2)}=  \dfrac{x + 1}{x - 2}
	\]

	\[
		\lim_{x \to 1^+} \dfrac{x + 1}{x - 2} = \dfrac{1 + 1}{1 - 2} = \dfrac{2}{-1} = -2.
	\]

	Dado que:
	\[ \lim_{x \to 1^-} f(x) = \lim_{x \to 1^+} f(x) = -2 \]

	$\implies$ Entonces el límite existe y es igual a \(-2\).\\

	(III). \textbf{Igualdad entre \( f(1) \) y \( \lim_{x \to 1} f(x) \):  }
	\[
		f(1) = -2 \quad \text{y} \quad \lim_{x \to 1} f(x) = -2.
	\]

	$\implies$ Dado que las tres condiciones de continuidad se satisfacen, entonces \textbf{\( f(x) \) es continua en \( x = 1 \)}.
	\newpage

	\question \large\textbf{Escribir las derivadasde las siguientes funciones:}

	\begin{parts}
		\part Para \( y = ax^n \), donde \( a \) es constante y \( n \in \mathbb{N} \):
		\[
			y' = a n x^{n-1}.
		\]

		\part Para \( y = \dfrac{a}{x} \), donde \( a \) es constante:
		\[
			y' = -\dfrac{a}{x^2}.
		\]

		\part Para \( y = a \cos(x) \), donde \( a \) es constante:
		\[
			y' = -a \sin(x).
		\]

		\part Para \( y = a b^x \), donde \( a \) y \( b \) son constantes y \( b > 1 \):
		\[
			y' = a b^x \ln(b).
		\]

		\part Para \( y = a \sin^{-1}(x) \), donde \( a \) es constante:
		\[
			y' = \dfrac{a}{\sqrt{1-x^2}}.
		\]

		\part Para \( y = \sin(x) \cos(x) \):
		Usamos el producto de funciones:
		\[
			y' = \cos^2(x) - \sin^2(x).
		\]

		\part Para \( y = \dfrac{\sin(x)}{\tan(x)} \):
		Simplificamos primero: \( \tan(x) = \dfrac{\sin(x)}{\cos(x)} \), entonces:
		\[
			y = \dfrac{\sin(x)}{\dfrac{\sin(x)}{\cos(x)}} = \cos(x).
		\]
		Por lo tanto:
		\[
			y' = -\sin(x).
		\]

		\part Para \( y = (\ln(a)) \log_a(x) \), donde \( a > 3 \):
		Usamos \( \log_a(x) = \dfrac{\ln(x)}{\ln(a)} \):
		\[
			y = \ln(a) \cdot \dfrac{\ln(x)}{\ln(a)} = \ln(x).
		\]
		Derivamos:
		\[
			y' = \dfrac{1}{x}.
		\]
	\end{parts}

	\newpage

	\question \large\textbf{Encontrar la derivada de las siguientes funciones:}


	\begin{parts}
		\part Para \( y = 2x^{-2} + 3e^2 - 2\ln(2x) \): \\

		Derivamos cada término:
		\[
			y' = \dfrac{d}{dx}\left( 2x^{-2} \right) + \dfrac{d}{dx}\left( 3e^2 \right) - \dfrac{d}{dx}\left( 2\ln(2x) \right).
		\]
		- Derivada del primer término:
		\[
			\dfrac{d}{dx}\left( 2x^{-2} \right) = -4x^{-3}.
		\]
		- Derivada del segundo término (\( 3e^2 \) es constante):
		\[
			\dfrac{d}{dx}\left( 3e^2 \right) = 0.
		\]
		- Derivada del tercer término (aplicamos regla de la cadena):
		\[
			\dfrac{d}{dx}\left( 2\ln(2x) \right) = 2 \cdot \dfrac{1}{2x} \cdot 2 = \dfrac{2}{x}.
		\]
		Entonces:
		\[
			y' = -4x^{-3} - \dfrac{2}{x}.
		\]

		\part Para \( y = \sqrt{\dfrac{2-x}{2+x}} \): \\

		Reescribimos la función como:
		\[
			y = \left( \dfrac{2-x}{2+x} \right)^{1/2}.
		\]
		Derivamos usando la regla de la cadena y la regla del cociente:

		\[
			y' = \dfrac{1}{2} \left( \dfrac{2-x}{2+x} \right)^{-1/2} \cdot \dfrac{d}{dx}\left( \dfrac{2-x}{2+x} \right).
		\]

		Calculamos la derivada del cociente :

		\[
			\dfrac{d}{dx}\left( \dfrac{2-x}{2+x} \right) = \dfrac{(2+x)(-1) - (2-x)(1)}{(2+x)^2} = \dfrac{-2 - x - 2 + x}{(2+x)^2} = \dfrac{-4}{(2+x)^2}.
		\]

		\[
			y' = \dfrac{1}{2} \left( \dfrac{2-x}{2+x} \right)^{-1/2} \cdot \dfrac{-4}{(2+x)^2}.
		\]

		\[
			y' = -\dfrac{2}{(2+x)^2 \sqrt{\dfrac{2-x}{2+x}}}.
		\]

		\[
			y' = -\dfrac{2 \sqrt{2+x}}{(2+x)^2 \sqrt{2-x}}.
		\]
	\end{parts}



	\question \large\textbf{Resolver:}
	\begin{parts}
		\part Encontrar $y'$ en la siguiente ecuación: $x-(y+1)^2=2xy$
		\part Encontrar la tercera derivada de $y=xe^{-x}$
	\end{parts}



	\begin{parts}
		\part Encontrar \( y' \) en la ecuación \( x - (y+1)^2 = 2xy \):\\

		Derivamos implícitamente con respecto a \( x \):
		\[
			\frac{d}{dx}\left( x \right) - \frac{d}{dx}\left( (y+1)^2 \right) = \frac{d}{dx}\left( 2xy \right).
		\]
		- Derivada del primer término:
		\[
			\frac{d}{dx}(x) = 1.
		\]
		- Derivada del segundo término (regla de la cadena):
		\[
			\frac{d}{dx}((y+1)^2) = 2(y+1)y'.
		\]
		- Derivada del tercer término (regla del producto):
		\[
			\frac{d}{dx}(2xy) = 2\left( y + xy' \right).
		\]
		Sustituimos:
		\[
			1 - 2(y+1)y' = 2y + 2xy'.
		\]
		Despejar \( y' \):
		\[
			1 - 2y - 2(y+1)y' = 2xy'.
		\]
		\[
			1 - 2y = 2xy' + 2(y+1)y'.
		\]
		\[
			1 - 2y = y'(2x + 2y + 2).
		\]
		\[
			y' = \frac{1 - 2y}{2x + 2y + 2}.
		\]

		\part Encontrar la tercera derivada de \( y = xe^{-x} \):\\

		Derivamos sucesivamente:
		- Primera derivada:
		\[
			y' = \frac{d}{dx}(xe^{-x}) = e^{-x} - xe^{-x}.
		\]
		- Segunda derivada:
		\[
			y'' = \frac{d}{dx}\left( e^{-x} - xe^{-x} \right) = -e^{-x} - (e^{-x} - xe^{-x}) = -2e^{-x} + xe^{-x}.
		\]
		- Tercera derivada:
		\[
			y''' = \frac{d}{dx}\left( -2e^{-x} + xe^{-x} \right) = 2e^{-x} + (e^{-x} - xe^{-x}).
		\]
		Simplificamos:
		\[
			y''' = 3e^{-x} - xe^{-x}.
		\]
	\end{parts}


	\newpage
	\question \large\textbf{La posicion en funcion del tiempo (distancia - tiempo) de un movil viene dado como: $x=-5t+10\sin(2t)$, donde $x$ esta en metros y $t$ esta en segundos.}
	\begin{parts}
		\part {Encuentre las ecuaciones de la velocidad(rapidez) y de la aceleracion.}
		\part {Encuentre la posicion, velocidad y aceleracion para los tiempos $t=0$ y $t=5$ segundos.}
	\end{parts}


	\begin{parts}
		\part Encuentre las ecuaciones de la velocidad (rapidez) y de la aceleración:
		La posición en función del tiempo está dada por:
		\[
			x(t) = -5t + 10\sin(2t).
		\]
		- La velocidad es la derivada de la posición con respecto al tiempo:
		\[
			v(t) = \frac{dx}{dt} = \frac{d}{dt} \left( -5t + 10\sin(2t) \right).
		\]
		Derivamos:
		\[
			v(t) = -5 + 10\cdot 2\cos(2t) = -5 + 20\cos(2t).
		\]
		- La aceleración es la derivada de la velocidad con respecto al tiempo:
		\[
			a(t) = \frac{dv}{dt} = \frac{d}{dt} \left( -5 + 20\cos(2t) \right).
		\]
		Derivamos:
		\[
			a(t) = -20 \cdot 2\sin(2t) = -40\sin(2t).
		\]

		\part Encuentre la posición, velocidad y aceleración para los tiempos \( t = 0 \) y \( t = 5 \) segundos:\\

		Sustituimos \( t = 0 \) y \( t = 5 \) en las ecuaciones:

		- Para \( t = 0 \):
		\[
			x(0) = -5(0) + 10\sin(2(0)) = 0,
		\]
		\[
			v(0) = -5 + 20\cos(2(0)) = -5 + 20(1) = 15,
		\]
		\[
			a(0) = -40\sin(2(0)) = -40(0) = 0.
		\]

		- Para \( t = 5 \):
		\[
			x(5) = -5(5) + 10\sin(2(5)) = -25 + 10\sin(10).
		\]
		\newpage
		Aproximamos \( \sin(10) \) usando radianes:
		\[
			x(5) \approx -25 + 10(-0.544) = -25 - 5.44 = -30.44.
		\]
		\[
			v(5) = -5 + 20\cos(10).
		\]
		Aproximamos \( \cos(10) \):
		\[
			v(5) \approx -5 + 20(-0.839) = -5 - 16.78 = -21.78.
		\]
		\[
			a(5) = -40\sin(10).
		\]
		Usando \( \sin(10) \approx -0.544 \):
		\[
			a(5) \approx -40(-0.544) = 21.76.
		\]

		Resultados:
		\begin{itemize}
			\item Para \( t = 0 \): \( x = 0 \), \( v = 15 \), \( a = 0 \).
			\item Para \( t = 5 \): \( x \approx -30.44 \), \( v \approx -21.78 \), \( a \approx 21.76 \).
		\end{itemize}
	\end{parts}



\end{questions}

\end{document}
