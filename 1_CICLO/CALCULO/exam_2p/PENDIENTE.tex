\documentclass[answers]{exam}
\usepackage[english]{babel}
\usepackage[letterpaper,top=2cm,bottom=2cm,left=3cm,right=3cm,marginparwidth=1.75cm]{geometry}
\usepackage{amsmath, amssymb}
\usepackage{graphicx}
\usepackage{enumitem}
\usepackage{tikz,pgfplots}
\pgfplotsset{compat=1.18}
\usepackage[colorlinks=true, allcolors=blue]{hyperref}

% Header and footer
\pagestyle{headandfoot}
\firstpageheader{}{}{}
\runningheader{}{}{}
\firstpagefooter{}{}{}
\runningfooter{}{}{}

\begin{document}

\begin{questions}

	% Question 1
	\question \large\textbf{Calcular limites:}
	\begin{enumerate}[label=\alph*.]
		\item $\displaystyle \lim_{x \to 0} \frac{\sqrt{25+x}-5}{\sqrt{1+x}-1} $
		\item $\displaystyle \lim_{x \to 0} \frac{2x^3 -3x}{2 \sin x} $
	\end{enumerate}
	\begin{solution}
		\begin{parts}
			\part Para resolverlo, multiplicamos el numerador y el denominador por las expresiones conjugadas correspondientes:

			\[
				\frac{\sqrt{25+x}-5}{\sqrt{1+x}-1} \cdot \frac{\sqrt{25+x}+5}{\sqrt{25+x}+5} \cdot \frac{\sqrt{1+x}+1}{\sqrt{1+x}+1}
			\]
	
			Esto nos da:
	
			\[
				\frac{(\sqrt{25+x})^2 - 5^2}{(\sqrt{1+x})^2 - 1^2} = \frac{25+x - 25}{1+x - 1} = \frac{x}{x}
			\]
	
			Entonces el límite es:
	
			\[
				\lim_{x \to 0} \frac{x}{x} = 1.
			\]
	
		

			\part Primero, intentamos sustituir directamente $x = 0$ en la expresión, pero obtenemos una forma indeterminada $\frac{0}{0}$, ya que $2x^3 - 3x = 0$ y $\sin(0) = 0$.

			Para resolverlo, podemos utilizar la expansión en serie de Taylor de $\sin x$ alrededor de $x = 0$. Sabemos que:
	
			\[
				\sin x = x - \frac{x^3}{6} + O(x^5)
			\]
	
			Sustituyendo esta aproximación en la expresión original:
	
			\[
				\frac{2x^3 - 3x}{2 \sin x} = \frac{2x^3 - 3x}{2\left(x - \frac{x^3}{6} + O(x^5)\right)}
			\]
	
			Factorizando el numerador:
	
			\[
				= \frac{x(2x^2 - 3)}{2\left(x - \frac{x^3}{6} + O(x^5)\right)}
			\]
	
			Ahora podemos simplificar $x$ en el numerador y denominador:
	
			\[
				= \frac{2x^2 - 3}{2\left(1 - \frac{x^2}{6} + O(x^4)\right)}
			\]
	
			Cuando $x \to 0$, el término $x^2$ en el denominador tiende a cero, por lo que la expresión se simplifica a:
	
			\[
				\lim_{x \to 0} \frac{2x^2 - 3}{2} = \frac{-3}{2}.
			\]
	
			Por lo tanto, el valor del límite es:
	
			\[
				\lim_{x \to 0} \frac{2x^3 - 3x}{2 \sin x} = -\frac{3}{2}.
			\]
			
		\end{parts}

	





	\end{solution}


	\question \large\textbf{Estudie continuidad de $f(x)=\dfrac{x^2-3}{x^2+2x-8}$}

	\begin{solution}
		\textbf{Solución:}
	
		Estudiamos la continuidad de \( f(x) = \frac{x^2 - 3}{x^2 + 2x - 8} \).
	
		**1. Asíntotas verticales:** 
	
		Factorizamos el denominador:
	
		\[
		x^2 + 2x - 8 = (x - 2)(x + 4)
		\]
	
		La función tiene potenciales discontinuidades en \( x = 2 \) y \( x = -4 \). Evaluamos el numerador en estos puntos: \\
	
		- Para \( x = 2 \), \( x^2 - 3 = 1 \neq 0 \).\\
		- Para \( x = -4 \), \( x^2 - 3 = 13 \neq 0 \).
	
		Dado que el numerador no se anula en estos puntos, no hay discontinuidades removibles. Por lo tanto, la función tiene discontinuidades no removibles en \( x = 2 \) y \( x = -4 \).
	
		**2. Conclusión:** La función es continua en \( \mathbb{R} \setminus \{2, -4\} \).
	\end{solution}
	
	
	\question \large\textbf{Determine las asintotas de la funcion $f(x)=\sqrt{4x^2+2x+1}$:}

	\begin{solution}
		\textbf{Solución:}
	
		Estudiamos las asíntotas de \( f(x) = \sqrt{4x^2 + 2x + 1} \).
	
		**1. Asíntotas verticales:**
	
		El radicando \( 4x^2 + 2x + 1 \) es siempre positivo (discriminante negativo), por lo que no hay asíntotas verticales.
	
		**2. Asíntotas horizontales:**
	
		La función tiende a \( \infty \) tanto para \( x \to \infty \) como para \( x \to -\infty \), por lo que no tiene asíntotas horizontales.
	
		**3. Asíntotas oblicuas:**\\
	
		- Para \( x \to \infty \), \( f(x) \sim 2x + \frac{1}{2} \).\\
		- Para \( x \to -\infty \), \( f(x) \sim -2x - \frac{1}{2} \).
	
		**Conclusión:** La función tiene asíntotas oblicuas en \( x \to \infty \) y \( x \to -\infty \), pero no tiene asíntotas verticales ni horizontales.
	\end{solution}
	
	
\end{questions}

\end{document}
