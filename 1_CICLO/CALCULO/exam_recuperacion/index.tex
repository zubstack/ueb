\documentclass[12pt, answers]{exam} % Clase para exámenes con respuestas
\usepackage[english,spanish]{babel} % Soporte para inglés y español
\usepackage[autostyle]{csquotes} % Manejo de citas
\usepackage{amsmath, amssymb} % Paquetes para matemáticas avanzadas
\usepackage{graphicx} % Inclusión de gráficos
\usepackage{enumitem} % Personalización de listas enumeradas
\usepackage[letterpaper,top=2cm,bottom=2cm,left=3cm,right=3cm,marginparwidth=1.75cm]{geometry} % Configuración de márgenes
\usepackage[colorlinks=true, allcolors=blue]{hyperref} % Enlaces con color

\renewcommand{\solutiontitle}{\noindent\textbf{Respuesta:}\par\noindent} % Personalización del título de respuestas
\renewcommand{\familydefault}{\sfdefault}

\begin{document}

\begin{center}
    \textbf{\large EXAMEN DE RECUPERACIÓN (CÁLCULO 1)}
\end{center}

\vspace{0.5cm}

\begin{questions}

    % Pregunta 1
    \question Sean las funciones \( f(x) = \ln(x + 1) \) y \( g(x) = x^2 - 5x + 6 \):
    \begin{parts}
        \part Encontrar matemáticamente el dominio y recorrido de cada función.
        \part Encontrar \(\frac{f(x)}{g(x)}\) y su dominio.
    \end{parts}

    % Pregunta 2
    \question Determinar si la siguiente función es continua en \( x = 1 \) y en \( x = 2 \); además graficarla a mano.
    \[
    f(x) = 
    \begin{cases} 
    x & \text{si } x < 1, \\
    x^2 + 1 & \text{si } 1 \leq x \leq 2, \\
    2 & \text{si } x > 2.
    \end{cases}
    \]

    % Pregunta 3
    \question Determine las derivadas solicitadas en los siguientes literales:
    % \begin{parts}
    %     \part La primera derivada de \( f(x) = \ln \left(\sqrt[3]{\frac{x-1}{x+1}}\right) \).
    %     \part La segunda derivada de \( f(x) = xe^x \).
    %     \part La tercera derivada de \( f(x) = \frac{1}{(x-a)^n} \) con \( a \) constante.
    %     \part La cuarta derivada de \( f(x) = \ln x^2 \).
    % \end{parts}

    \begin{solution}

        \begin{parts}
            \part La primera derivada de \( f(x) = \ln \left(\sqrt[3]{\frac{x-1}{x+1}}\right) \):
            \[
            f(x) = \ln \left(\left(\frac{x-1}{x+1}\right)^{\frac{1}{3}}\right)
            \]
            \[
            f(x) = \frac{1}{3} \ln \left(\frac{x-1}{x+1}\right)
            \]
            \[
            f'(x) = \frac{1}{3} \cdot \frac{1}{\frac{x-1}{x+1}} \cdot \frac{d}{dx}\left(\frac{x-1}{x+1}\right)
            \]
            \[
            \frac{d}{dx}\left(\frac{x-1}{x+1}\right) = \frac{(x+1)\cdot 1 - (x-1)\cdot 1}{(x+1)^2}
            \]
            \[
            \frac{d}{dx}\left(\frac{x-1}{x+1}\right) = \frac{x+1-(x-1)}{(x+1)^2} = \frac{2}{(x+1)^2}
            \]
            \[
            f'(x) = \frac{1}{3} \cdot \frac{x+1}{x-1} \cdot \frac{2}{(x+1)^2} = \frac{2}{3(x^2-1)}
            \]
    
            \part La segunda derivada de \( f(x) = xe^x \):
            \[
            f'(x) = \frac{d}{dx}\left(xe^x\right)
            \]
            \[
            f'(x) = x \cdot \frac{d}{dx}\left(e^x\right) + e^x \cdot \frac{d}{dx}(x)
            \]
            \[
            f'(x) = xe^x + e^x
            \]
            \[
            f''(x) = \frac{d}{dx}\left(xe^x + e^x\right)
            \]
            \[
            f''(x) = x \cdot \frac{d}{dx}\left(e^x\right) + e^x \cdot \frac{d}{dx}(x) + \frac{d}{dx}\left(e^x\right)
            \]
            \[
            f''(x) = xe^x + e^x + e^x = e^x(x + 2)
            \]
    
            \part La tercera derivada de \( f(x) = \frac{1}{(x-a)^n} \):
            \[
            f(x) = (x-a)^{-n}
            \]
            \[
            f'(x) = -n(x-a)^{-n-1} \cdot \frac{d}{dx}(x-a)
            \]
            \[
            f'(x) = -n(x-a)^{-n-1}
            \]
            \[
            f''(x) = \frac{d}{dx}\left(-n(x-a)^{-n-1}\right)
            \]
            \[
            f''(x) = -n(-n-1)(x-a)^{-n-2}
            \]
            \[
            f''(x) = n(n+1)(x-a)^{-n-2}
            \]
            \[
            f'''(x) = \frac{d}{dx}\left(n(n+1)(x-a)^{-n-2}\right)
            \]
            \[
            f'''(x) = n(n+1)(-n-2)(x-a)^{-n-3}
            \]
            \[
            f'''(x) = -n(n+1)(n+2)(x-a)^{-n-3}
            \]
    
            \part La cuarta derivada de \( f(x) = \ln x^2 \):
            \[
            f(x) = 2 \ln x
            \]
            \[
            f'(x) = 2 \cdot \frac{1}{x}
            \]
            \[
            f'(x) = \frac{2}{x}
            \]
            \[
            f''(x) = \frac{d}{dx}\left(\frac{2}{x}\right)
            \]
            \[
            f''(x) = 2 \cdot \frac{d}{dx}\left(x^{-1}\right)
            \]
            \[
            f''(x) = 2(-1)x^{-2} = \frac{-2}{x^2}
            \]
            \[
            f'''(x) = \frac{d}{dx}\left(\frac{-2}{x^2}\right)
            \]
            \[
            f'''(x) = -2(-2)x^{-3} = \frac{4}{x^3}
            \]
            \[
            f^{(4)}(x) = \frac{d}{dx}\left(\frac{4}{x^3}\right)
            \]
            \[
            f^{(4)}(x) = 4(-3)x^{-4} = \frac{-12}{x^4}
            \]
        \end{parts}
    
    \end{solution}
    
    

    % Pregunta 4
    \question Determine el ángulo que forma la curva \( y = e^{0.5x} \) con la recta \( x = 2 \) en el punto de intersección.

    \begin{solution}

        Para encontrar el ángulo que forma la curva \( y = e^{0.5x} \) con la recta \( x = 2 \):
    
        1. **Derivada de la función:**
        \[
        y' = \frac{d}{dx}\left(e^{0.5x}\right)
        \]
        \[
        y' = 0.5e^{0.5x}
        \]
        \[
        y'(2) = 0.5e^{0.5 \times 2} = 0.5e
        \]
    
        2. **Cálculo del ángulo:**
        \[
        \theta = \tan^{-1}(0.5e)
        \]
    
    \end{solution}
    
    

\end{questions}

\end{document}
