\documentclass[answers]{exam} % Clase para exámenes con respuestas
\usepackage[english,spanish]{babel} % Soporte para inglés y español
\usepackage[autostyle]{csquotes} % Manejo de citas
\usepackage{amsmath, amssymb} % Paquetes para matemáticas avanzadas
\usepackage{graphicx} % Inclusión de gráficos
\usepackage{enumitem} % Personalización de listas enumeradas
\usepackage[letterpaper,top=2cm,bottom=2cm,left=3cm,right=3cm,marginparwidth=1.75cm]{geometry} % Configuración de márgenes
\usepackage[colorlinks=true, allcolors=blue]{hyperref} % Enlaces con color

\renewcommand{\solutiontitle}{\noindent\textbf{Respuesta:}\par\noindent} % Personalización del título de respuestas
\renewcommand{\familydefault}{\sfdefault}

% Configuración de encabezado y pie de página
\pagestyle{headandfoot}
\firstpageheader{Universidad de Bolívar}{}{17 de julio del 2024} 
\runningheader{Universidad de Bolívar}{}{Calculo I}
\firstpagefooter{}{\thepage}{}
\runningfooter{}{\thepage}{}

\begin{document}

\begin{center}
    \large\textbf{Trabajo Autónomo 2.13 - Cálculo I}\\[1em]
    \large Primer Ciclo \enquote*{A} - Ingeniería de Software\\[1em]
\end{center}

\vspace{0.5cm}
\large\textbf{Estudiante:} Ariel Alejandro Calderón
\vspace{0.5cm}

Resolver todos los ejercicios propuestos de cada tema presentado en esta semana.

\begin{questions}

    % Question 1
    \question \large\textbf{Encontrar las derivadas de las siguientes funciones:}
    \begin{enumerate}[label=\alph*.]
        \item $\displaystyle f(x)=2x^2-3x+4$
        \item $\displaystyle f(x)=cos^2(x)$
        \item $\displaystyle f(x)=x\sqrt[3]{x}$
        \item $\displaystyle f(x)=\dfrac{x^2-1}{x^2+1}$
        \item $\displaystyle f(x)=x*cos(x)$
        \item $\displaystyle f(x)=\dfrac{sin(x)}{x}$
    \end{enumerate}
    \begin{solution}
        \begin{enumerate}[label=\alph*.]
            \item $\displaystyle f'(x)  = 4x - 3$ \\
            \item $\displaystyle f'(x)  = 2\cos(x)(-\sin(x)) = -2\cos(x)\sin(x)$ \\
            \item $\displaystyle f'(x)  = x^{1/3} + \frac{1}{3}x^{1/3-1}x = x^{1/3} + \frac{1}{3}x^{-2/3}x = x^{1/3} + \frac{1}{3}x^{-1/3} = \frac{4}{3}x^{1/3}$\\
            \item $\displaystyle f'(x) = \frac{(2x)(x^2 + 1) - (x^2 - 1)(2x)}{(x^2 + 1)^2} = \frac{2x(x^2 + 1 - x^2 + 1)}{(x^2 + 1)^2} \\ = \frac{2x(1 + 1)}{(x^2 + 1)^2} = \frac{4x}{(x^2 + 1)^2}$\\
            \item $\displaystyle f'(x) = \cos(x) - x \sin(x)$\\
            \item $\displaystyle f'(x) =  \frac{x \cos(x) - \sin(x)}{x^2} = \frac{x \cos(x) - \sin(x)}{x^2}$\\
        \end{enumerate}
    \end{solution}
    \vspace{0.5cm}
    \newpage
    % Question 2
    \question \large\textbf{Encontrar las derivadas de las siguientes funciones:}
    \begin{enumerate}[label=\alph*.]
        \item $\displaystyle f(x)=\sqrt{x^3-2x^2+x-2}$
        \item $\displaystyle f(x)=\sec\left(\dfrac{x-1}{x+1}\right)$
        \item $\displaystyle f(x)=\sqrt{\sqrt{\sqrt{x^3+x-2}}}$
        \item $\displaystyle f(x)=\ln(x+1)^2$
    \end{enumerate}
    \begin{solution}
        \begin{enumerate}[label=\alph*.]
            \item $\displaystyle f'(x) =  \frac{1}{2\sqrt{x^3-2x^2+x-2}} \cdot (3x^2-4x+1)$\\
            \item $\displaystyle f'(x) =  \sec\left(\frac{x-1}{x+1}\right) \tan\left(\frac{x-1}{x+1}\right) \cdot \frac{(x+1) - (x-1)}{(x+1)^2}\\ = \sec\left(\frac{x-1}{x+1}\right) \tan\left(\frac{x-1}{x+1}\right) \cdot \frac{2}{(x+1)^2}$\\
            \item $\displaystyle f'(x) =  \frac{1}{6} (x^3+x-2)^{-5/6} \cdot (3x^2+1)$\\
            \item $\displaystyle f'(x) = \frac{2\ln(x+1)}{x+1}$\\
        \end{enumerate}
    \end{solution}
    

    \vspace{0.5cm}

    % Question 3
    \question \large\textbf{Encontrar las derivadas de las siguientes funciones:}
    \begin{enumerate}[label=\alph*.]
        \item $\displaystyle y=sen^{-1}(x^2)$
        \item $\displaystyle y=tg^{-1}\sqrt{x^2+2}$
        \item $\displaystyle \sin x = x (1 + \tan y)$
        \item $\displaystyle y=cos(xy)$
    \end{enumerate}
    \begin{solution}
        \begin{enumerate}[label=\alph*.]
            \item $\displaystyle y' = \frac{2x}{\sqrt{1-(x^2)^2}} = \frac{2x}{\sqrt{1-x^4}}$
            \item $\displaystyle y' = \frac{x}{(x^2+3)\sqrt{x^2+2}}$
            \item $\displaystyle \cos x = x (1 + \tan y)$
            
            Derivando ambos lados respecto a \(x\):
            \[
            -\sin x = (1 + \tan y) + x \sec^2 y y'
            \]
            \[
            x \sec^2 y y' = -\sin x - 1 - \tan y
            \]
            \[
            y' = \frac{-\sin x - 1 - \tan y}{x \sec^2 y}
            \]
            \item $\displaystyle y' = -\frac{y \sin(xy)}{x \sin(xy) + \cos(xy)}$
        \end{enumerate}
    \end{solution}
    
    

    \vspace{0.5cm}

\end{questions}

\end{document}
